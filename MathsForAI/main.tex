\documentclass[12pt,french]{report}

% This is the main framework I use for my LateX documents.
% main.tex by Alexis GRACIAS

%%%%%%%%%%%%
% PACKAGES %
%%%%%%%%%%%%

% TABLE OF CONTENT FOR CHAPTERS

\usepackage{minitoc}
\usepackage{minitoc}
\mtcsettitle{minitoc}{Sommaire du chapitre} % Change le titre de la minitoc
\mtcsetfont{minitoc}{section}{\normalsize\bfseries} % Style des sections dans la minitoc
\mtcsetfont{minitoc}{subsection}{\small\itshape} % Style des sous-sections
\mtcsetrules{minitoc}{off} % Désactive les lignes sous les titres de section
\mtcsetoffset{minitoc}{10pt} % Ajoute un espacement autour de la minitoc

\dominitoc % Active les mini-tables des matières
% Personnalisation du titre des mini tables des matières
\renewcommand{\mtctitle}{}


% GEOMETRY

\usepackage[left=2cm,right=2cm,top=2cm,bottom=2cm]{geometry}

% LASTAPGE

\usepackage{lastpage}

% HEADERS AND FOOTERS
\usepackage{fancyhdr}

% LANGUAGE

\usepackage{babel}

% REFERENCES

\usepackage{hyperref} % Add links to the table of content, files and website

% GRAPHS

\usepackage{tikz} % Graph package

% IMAGES

\usepackage{graphicx} % Required for inserting images
\usepackage{tabto}

% COLOR

\usepackage[skins,theorems]{tcolorbox} % Frame color package
\usepackage{amssymb}
\usepackage[dvipsnames]{xcolor} % Text color package and more colors
\usepackage{tcolorbox}

% Custom colors
\definecolor{ao}{rgb}{0.0, 0.0, 1.0}
\definecolor{coolblack}{rgb}{0.0, 0.18, 0.39}
\definecolor{cyan}{rgb}{0.0, 1.0, 1.0}
\definecolor{glaucous}{rgb}{0.38, 0.51, 0.71}
\definecolor{electricultramarine}{rgb}{0.25, 0.0, 1.0}

% CAPTIONS

\usepackage{caption}
\usepackage{subcaption}
\usepackage[utf8]{inputenc}

% STYLE

% Chapter's title style
\usepackage[T1]{fontenc}
\usepackage{titlesec, blindtext, color}
\definecolor{gray75}{gray}{0.75}
\newcommand{\hsp}{\hspace{20pt}}
\titleformat{\chapter}[hang]{\Huge\bfseries}{\thechapter\hsp\textcolor{gray75}{|}\hsp}{0pt}{\Huge\bfseries}

% Table of contents style
\makeatletter
\renewcommand*{\l@chapter}[2]{%         % Chapter label format
  \ifnum \c@tocdepth > \m@ne            % Check: Maximum depth > -1
    \addpenalty{-\@highpenalty}%        % Encourage prior line breaking
    \vskip 1.0em \@plus\p@%             % Prior vertical whitespace
    \setlength\@tempdima{1.5em}%        % Set temporary length
    \begingroup                         % Begin closed group
      \parindent\z@                     % No paragraph indentation
      \rightskip\@tocrmarg              % Reserved right whitespace
      \parfillskip -\@tocrmarg          % Prevent box from moving
      \leavevmode                       % Force horizontal mode
      \advance\leftskip\@tempdima       % Left whitespace
      \hskip -\leftskip                 % First line alignment
      {\bfseries #1}%           % Chapter name and format
      \nobreak                          % Avoid lineskip
      \leaders\hbox{$%                  % Command to replicate dots
        \m@th                           % Math mode for alignment
        \mkern\@dotsep mu               % Whitespace before a dot
        \hbox{.}%                       % Dot separator
        \mkern\@dotsep mu               % Whitespace after a dot
      $}%
      \hfill                            % Flush to the right
      \nobreak\hb@xt@\@pnumwidth{%      % Chapter page’s number box
        \hss #2%                        % Page number with buffer
      }%
      \par                              % Insert line skip
      \penalty\@highpenalty             % Return to default penalty
    \endgroup                           % End closed group
  \fi
}
\makeatother


% MATH

\usepackage{amsmath,amsfonts,mathtools,stmaryrd} % Math libraries

\usepackage{listings} % required for specific languages
\lstset{ % Set listing package options
    language=bash, % choose the language of the code
    basicstyle=\fontfamily{pcr}\selectfont\footnotesize\color{red},
    keywordstyle=\color{black}\bfseries, % style for keywords
    numbers=none, % where to put the line-numbers
    numberstyle=\tiny, % the size of the fonts that are used for the line-numbers     
    backgroundcolor=\color{white},
    showspaces=false, % show spaces adding particular underscores
    showstringspaces=false, % underline spaces within strings
    showtabs=false, % show tabs within strings adding particular underscores
    frame=single, % adds a frame around the code
    tabsize=2, % sets default tabsize to 2 spaces
    rulesepcolor=\color{gray},
    rulecolor=\color{black},
    captionpos=b, % sets the caption-position to bottom
    breaklines=true, % sets automatic line breaking
    breakatwhitespace=false, 
}

% Tabular

\usepackage{tabularx}
\usepackage{ragged2e}

\newcolumntype{L}{>{\RaggedRight\arraybackslash}X} 
\newcolumntype{C}{>{\centering\arraybackslash}X}

% OTHER

\usepackage{pdfpages}

% SCRIPTS (python)

% Load libraries
\usepackage{tcolorbox}
\tcbuselibrary{minted,breakable,xparse,skins}
\usepackage{minted}
\usepackage{xcolor}

\definecolor{bg}{gray}{0.95}
\DeclareTCBListing{mintedbox}{O{}m!O{}}{%
  breakable=true,
  listing engine=minted,
  listing only,
  minted language=#2,
  minted style=default,
  minted options={%
    linenos,
    gobble=0,
    breaklines=true,
    breakafter=,,
    fontsize=\small,
    numbersep=8pt,
    #1},
  boxsep=0pt,
  left skip=0pt,
  right skip=0pt,
  left=25pt,
  right=0pt,
  top=3pt,
  bottom=3pt,
  arc=5pt,
  leftrule=0pt,
  rightrule=0pt,
  bottomrule=2pt,
  toprule=2pt,
  colback=bg,
  colframe=gray!70,
  enhanced,
  overlay={%
    \begin{tcbclipinterior}
    \fill[gray!20!white] (frame.south west) rectangle ([xshift=20pt]frame.north west);
    \end{tcbclipinterior}},
  #3}

%\usepackage{eso-pic,lipsum}
%\AddToShipoutPicture{%
%	\AtTextCenter{%
%		\fboxsep5mm \fboxrule=0.8pt
%		\makebox(0,0)[c]{\fbox{\rule{0pt}\textheight\rule\textwidth{0pt}}}%
%	}%
%}

%%%%%%%%%%%%
% COMMANDS %
%%%%%%%%%%%%

\lstset{aboveskip=\baselineskip,belowskip=\baselineskip,basicstyle=\ttfamily} % Formating line break after bash commands

% Get rid of 0. chapter's number
\makeatletter 
\renewcommand{\thesection}{%
  \ifnum\c@chapter<1 \@arabic\c@section
  \else \thechapter.\@arabic\c@section
  \fi
}
\makeatother

% Rules for \bar{x} and \overline{x} commands
\makeatletter
\newcommand*{\Xbar}{}%
\DeclareRobustCommand*{\Xbar}{%
  \mathpalette\@Xbar{}%
}


% THEOREMS

\definecolor{electricultramarine}{rgb}{0.25, 0.0, 1.0}

\tcbuselibrary{theorems} 
\newtcbtheorem
  []% init options
  {Syntax}% name
  {Syntax}% title
  {%
    colback=BrickRed!2,
    colframe=BrickRed!35!BrickRed,
    fonttitle=\bfseries,
  }% options
  {def}% prefix

\tcbuselibrary{theorems} 
\newtcbtheorem
  []% init options
  {Input}% name
  {Input}% title
  {%
    colback=BlueViolet!5,
    colframe=BlueViolet!35!BlueViolet,
    fonttitle=\bfseries,
  }% options
  {def}% prefix

\tcbuselibrary{theorems} 
\newtcbtheorem
  []% init options
  {Output}% name
  {Output}% title
  {%
    colback=Emerald!5,
    colframe=Emerald!35!Emerald,
    fonttitle=\bfseries,
  }% options
  {def}% prefix

% FORMULAS

\tcbset{highlight math style={enhanced,
  colframe=red,colback=white,arc=0pt,boxrule=1pt}}

% CUSTOM COMMANDS

\newcommand*{\@Xbar}[2]{%
  % #1: math style
  % #2: unused (empty)
  \sbox0{$#1\mathrm{X}\m@th$}%
  \sbox2{$#1X\m@th$}%
  \rlap{%
    \hbox to\wd2{%
      \hfill
      $\overline{%
        \vrule width 0pt height\ht0 %
        \kern\wd0 %
      }$%
    }%
  }%
  \copy2 %
}
\makeatother

%%%%%%%%%%%%%%%%%%%%%%%%%%%%%%
% DECLARE UNICODE CHARACTERS %
%%%%%%%%%%%%%%%%%%%%%%%%%%%%%%

% Main
\DeclareUnicodeCharacter{2500}{\textendash}  % ─
\DeclareUnicodeCharacter{2502}{\textbar}     % │
\DeclareUnicodeCharacter{250C}{\textup{┌}}   % ┌
\DeclareUnicodeCharacter{2510}{\textup{┐}}   % ┐
\DeclareUnicodeCharacter{2514}{\textup{└}}   % └
\DeclareUnicodeCharacter{2518}{\textup{┘}}   % ┘
\DeclareUnicodeCharacter{251C}{\textup{├}}   % ├
\DeclareUnicodeCharacter{2524}{\textup{┤}}   % ┤
\DeclareUnicodeCharacter{252C}{\textup{┬}}   % ┬
\DeclareUnicodeCharacter{2534}{\textup{┴}}   % ┴
\DeclareUnicodeCharacter{253C}{\textup{┼}}   % ┼
\DeclareUnicodeCharacter{2550}{\textup{═}}   % ═
\DeclareUnicodeCharacter{2551}{\textup{║}}   % ║
\DeclareUnicodeCharacter{2552}{\textup{╒}}   % ╒
\DeclareUnicodeCharacter{2553}{\textup{╓}}   % ╓
\DeclareUnicodeCharacter{2554}{\textup{╔}}   % ╔
\DeclareUnicodeCharacter{2555}{\textup{╕}}   % ╕
\DeclareUnicodeCharacter{2556}{\textup{╖}}   % ╖
\DeclareUnicodeCharacter{2557}{\textup{╗}}   % ╗
\DeclareUnicodeCharacter{2558}{\textup{╘}}   % ╘
\DeclareUnicodeCharacter{2559}{\textup{╙}}   % ╙
\DeclareUnicodeCharacter{255A}{\textup{╚}}   % ╚
\DeclareUnicodeCharacter{255B}{\textup{╛}}   % ╛
\DeclareUnicodeCharacter{255C}{\textup{╜}}   % ╜
\DeclareUnicodeCharacter{255D}{\textup{╝}}   % ╝
\DeclareUnicodeCharacter{2560}{\textup{╠}}   % ╠
\DeclareUnicodeCharacter{2561}{\textup{╡}}   % ╡
\DeclareUnicodeCharacter{2562}{\textup{╢}}   % ╢
\DeclareUnicodeCharacter{2563}{\textup{╣}}   % ╣
\DeclareUnicodeCharacter{2564}{\textup{╤}}   % ╤
\DeclareUnicodeCharacter{2565}{\textup{╥}}   % ╥
\DeclareUnicodeCharacter{2566}{\textup{╦}}   % ╦
\DeclareUnicodeCharacter{2567}{\textup{╧}}   % ╧
\DeclareUnicodeCharacter{2568}{\textup{╨}}   % ╨
\DeclareUnicodeCharacter{2569}{\textup{╩}}   % ╩
\DeclareUnicodeCharacter{256A}{\textup{╪}}   % ╪
\DeclareUnicodeCharacter{256B}{\textup{╫}}   % ╫
\DeclareUnicodeCharacter{256C}{\textup{╬}}   % ╬

% Blocks and bars
\DeclareUnicodeCharacter{2580}{\textup{▀}}   % ▀
\DeclareUnicodeCharacter{2584}{\textup{▄}}   % ▄
\DeclareUnicodeCharacter{2588}{\textup{█}}   % █
\DeclareUnicodeCharacter{258C}{\textup{▌}}   % ▌
\DeclareUnicodeCharacter{2590}{\textup{▐}}   % ▐
\DeclareUnicodeCharacter{2591}{\textup{░}}   % ░
\DeclareUnicodeCharacter{2592}{\textup{▒}}   % ▒
\DeclareUnicodeCharacter{2593}{\textup{▓}}   % ▓
\DeclareUnicodeCharacter{25A0}{\textup{■}}   % ■
\DeclareUnicodeCharacter{25A1}{\textup{□}}   % □
\DeclareUnicodeCharacter{25AA}{\textup{▪}}   % ▪
\DeclareUnicodeCharacter{25AB}{\textup{▫}}   % ▫

% Geometry
\DeclareUnicodeCharacter{25B6}{\textup{▶}}   % ▶
\DeclareUnicodeCharacter{25C0}{\textup{◁}}   % ◁
\DeclareUnicodeCharacter{25FB}{\textup{▫}}   % ▫
\DeclareUnicodeCharacter{25FD}{\textup{▽}}   % ▽
\DeclareUnicodeCharacter{25FE}{\textup{▾}}   % ▾

% Currency
\DeclareUnicodeCharacter{20AC}{\textup{€}}   % €
\DeclareUnicodeCharacter{00A3}{\textup{£}}   % £
\DeclareUnicodeCharacter{20B9}{\textup{₹}}   % ₹

% Arrows
\DeclareUnicodeCharacter{2190}{\textup{←}}   % ←
\DeclareUnicodeCharacter{2191}{\textup{↑}}   % ↑
\DeclareUnicodeCharacter{2192}{\textup{→}}   % →
\DeclareUnicodeCharacter{2193}{\textup{↓}}   % ↓
\DeclareUnicodeCharacter{21D0}{\textup{⇐}}   % ⇐
\DeclareUnicodeCharacter{21D1}{\textup{⇑}}   % ⇑
\DeclareUnicodeCharacter{21D2}{\textup{⇒}}   % ⇒
\DeclareUnicodeCharacter{21D3}{\textup{⇓}}   % ⇓

% Miscellaneous
\DeclareUnicodeCharacter{2605}{\textup{★}}   % ★
\DeclareUnicodeCharacter{2606}{\textup{☆}}   % ☆
\DeclareUnicodeCharacter{2610}{\textup{☐}}   % ☐
\DeclareUnicodeCharacter{2611}{\textup{☑}}   % ☑
\DeclareUnicodeCharacter{2612}{\textup{☒}}   % ☒
\DeclareUnicodeCharacter{2620}{\textup{☠}}   % ☠
\DeclareUnicodeCharacter{2622}{\textup{☢}}   % ☢
\DeclareUnicodeCharacter{2623}{\textup{☣}}   % ☣
\DeclareUnicodeCharacter{2626}{\textup{☦}}   % ☦
\DeclareUnicodeCharacter{262A}{\textup{☪}}   % ☪
\DeclareUnicodeCharacter{262F}{\textup{☯}}   % ☯
\DeclareUnicodeCharacter{2660}{\textup{♠}}   % ♠
\DeclareUnicodeCharacter{2665}{\textup{♥}}   % ♥
\DeclareUnicodeCharacter{2666}{\textup{♦}}   % ♦
\DeclareUnicodeCharacter{2663}{\textup{♣}}   % ♣

%%%%%%%%%%%%%%%%%%%%%%
% DOCUMENTS SETTINGS %
%%%%%%%%%%%%%%%%%%%%%%

% Fist page
\newcommand{\HRule}{\rule{\linewidth}{0.5mm}}

\title{
    \HRule \\[0.4cm] \Huge \bfseries 
    Machine learning cheatsheet 
    \\[0.15cm] \HRule \\[0.4cm]}
\author{\Large Alexis GRACIAS}


% Document
\begin{document}


\maketitle

\dominitoc % Initialization
\large \tableofcontents
\normalsize % Text size

% Headers and footers
\renewcommand{\chaptermark}[1]{ \markboth{#1}{} }
\pagestyle{fancy}
%... then configure it.
\fancyhead{} % clear all header fields
\fancyhead[RO,LE]{\leftmark}
\fancyhead[LO,LE]{A. GRACIAS}
\fancyfoot{} % clear all footer fields

% Page number
\makeatletter
\renewcommand{\@evenfoot}{\makebox[\textwidth][c]{page {\thepage} sur \pageref{LastPage}}}
\renewcommand{\@oddfoot}{\@evenfoot}
\makeatother

% ITEMIZE (lists)
\renewcommand{\labelitemi}{$\bullet$}
\renewcommand{\labelitemii}{---}
\renewcommand{\labelitemiii}{-}
\renewcommand{\labelitemiv}{.}

%%%%%%%%%%%
% INCLUDE %
%%%%%%%%%%%

\chapter{Formulation mathématique d'un \textit{dataset}}


\newpage

\chapter{Métriques}
\section{Régression}

\begin{itemize}

    \item \textbf{Mean Absolute Error} : 
        \begin{equation}
            MSE(Y) = \frac{1}{n}\sum_{i = 1}^{n}|y_{i} - \hat{y_{i}}|
        \end{equation} \\
    \item \textbf{Root Mean Squarred Error} : 
        \begin{equation}
            MSE(Y) = \frac{1}{n}\sum_{i = 1}^{n}(y_{i} - \hat{y_{i}})^{2}
        \end{equation} \\

    \item 
\end{itemize}

\section{Classification}

\begin{itemize}

    \item \textbf{Accuracy score} : 
        \begin{equation}
            \text{Accuracy} = \frac{VP + VN}{VP + VN + FN + FP} = 1 - exactitude
        \end{equation} 

    Avec : \newline

    \begin{itemize}
        \item  VP : Vrai négatif   \\
        \item  FP : Faux positifs  \\
        \item  FN : Faux négatifs  \\
        \item  VP : Vrais positifs \\
    \end{itemize} 

    \item \textbf{Matrice de confusion} \newline
    Une matrice de confusion recense le nombre de faux positifs, faux négatifs, vrai positifs et vrai négatifs 
    sous forme de tableau. Les éléments de cette matrice $2 \times 2$ veulent dire ceci : \newline

    \begin{itemize}
        \item  Vrai négatif (Réel 0, Prédit 0)   \\
        \item  Faux positifs (Réel 0, Prédit 1)  \\
        \item  Faux négatifs (Réel 1, Prédit 0)  \\ 
        \item  Vrais positifs (Réel 1, Prédit 1) \\
    \end{itemize}

    La prédiction est la valeur que le modèle prédit tandis que réel est la valeur réelle, qui peut être observée à *posteriori*
    Par exemple, une matrice de confusion peut ressembler à ça : \newline


    \begin{table}[h!]
    \scriptsize
    \begin{tabularx}{\textwidth}{|C|C|C|}
    \hline
      &\textbf{Prédit : 0}   & \textbf{Prédit : 1}  \\
    \hline
    Réel : 0 & 87 & 13 \\
    \hline
    Réel : 1 & 18 & 61 \\
    \hline
    \end{tabularx}
    \caption{Exemple de matrice de confusion}
    \end{table}

    Ici, le modèle a prédit que 87 + 18 = 105 personnes n'ont pas survécues à l'accident sur les 87 + 13 = 100 personnes qui n'ont réellement pas survécues.
    De même, le modèle prédit que 13 + 61 = 74 personnes on survécues sur 10 + 61 = 71 qui ont réelement survécues. \\

\end{itemize}

\newpage

\chapter{Synthèse des modèles et leurs usages}

\begin{table}[h!]
\scriptsize
\begin{tabularx}{\textwidth}{|C|C|C|C|C|C|C|}
\hline
\textbf{Modèle}                                    & \textbf{Quand l'utiliser}                                                                 & \textbf{Paradigme}                   & \textbf{Formule}                                                                & \textbf{Quantité à minimiser}                                                                      & \textbf{Préparation des données}  \\
\hline

Régression linéaire                                & Relation linéaire entre $X_i$ et $X_j$                                                    & Régression statistique               & $\hat{y} = X\beta + \varepsilon$                                                & $||y - X\beta||^2$                                                                                 & Standardisation \\
\hline

Régression Ridge                                   & Régression avec pénalisation $L_2$ (sélection de variables)                               & Régression statistique               & $\hat{y} = X\beta + \varepsilon$                                                & $||y-X\beta||^2 + \lambda ||\beta||^2$                                                             & Standardisation \\
\hline

Régression Lasso                                   & Régression avec pénalisation $L_1$ (sélection de variables avec annulation)               & Régression statistique               & $\hat{y} = X\beta + \varepsilon$                                                & $||y-X\beta||^2 + \lambda ||\beta||_1$                                                             & Standardisation \\
\hline

Elastic Net                                        & Régression Ridge + Lasso                                                                  & Régression statistique               & $\hat{y} = X\beta + \varepsilon$                                                & $||y-X\beta||^2 + \lambda(\alpha ||\beta||_1 +(1-\alpha)||\beta||_2^2)$                            & Standardisation \\
\hline

Régression logistique                              & Classification binaire                                                                    & Classification binaire               & $\hat{y} = \sigma(X\beta)$                                                      & $-\sum y\log(\hat y)$                                                                              & Standardisation \\
\hline

Régression logistique multinomiale                 & Classification                                                                            & Classification                       &                                                                                 &                                                                                                    & Standardisation \\
\hline

Régression polynomiale                             & Régression non linéaire                                                                   & Régression statistique               & $\hat{y} = \sum_{k=0}^{d} \beta_k x^k$                                          & $\sum_{i=1}^{n}(y_i - \hat{y_i})^2$                                                                & Standardisation \\
\hline

Régression splines                                 & Régression non linéaire, robuste, locale                                                  & Régression statistique               &                                                                                 &                                                                                                    & -               \\
\hline

Régression additive généralisée (GAM)              & Régression fortement non linéaire, robuste, locale                                        & Régression statistique               &                                                                                 &                                                                                                    & -               \\
\hline

Régression isotonique                              & Régression non linéaire monotone                                                          & Régression statistique               &                                                                                 &                                                                                                    & Standardisation \\
\hline

Régression Généralisée (GLM)                       & Régression non linéaire                                                                   & Régression statistique               &                                                                                 &                                                                                                    & Standardisation \\
\hline

Quantile regression                                & Régression fortement non linéaire, robuste, présence d'hétéroscédasticité                 & Régression statistique               &                                                                                 & pinball loss                                                                                       & Standardisation \\
\hline

Processus gaussiens (GP)                           & Régression fortement non linéaire                                                         & Régression statistique               &                                                                                 &                                                                                                    & - \\
\hline

Huber regression                                   & Régression fortement non linéaire, robuste, insensible aux outliers                       & Régression statistique               &                                                                                 & huber loss                                                                                         & - \\
\hline

Régression PLS                                     & Régression non linéaire avec analyse des composantes principales                          & Régression non supervisée            &                                                                                 &                                                                                                    & Standardisation \\
\hline

KNN regression                                     & Régression locale / ponctuelle avec beaucoup de données                                   & Régression statistique               & $\hat{y} = \frac{1}{k}\sum_{i \in \mathcal{N}_k(x)} y_i$                        & $\sum_{i=1}^{n}(y_i - \hat{y_i})^2$                                                                & Standardisation \\
\hline

\end{tabularx}
\caption{Synthèse des modèles de Machine Learning - régression}
\end{table}

\newpage

\begin{table}[h!]
\scriptsize
\begin{tabularx}{\textwidth}{|C|C|C|C|C|C|C|}
\hline
\textbf{Modèle}                                    & \textbf{Quand l'utiliser}                                                                 & \textbf{Paradigme}                   & \textbf{Formule}                                                                & \textbf{Quantité à minimiser}                                                                      & \textbf{Préparation des données}  \\
\hline

SVM (Support Vector Machine)                       & Classification fortement non linéaire à marge dure                                        & Classification supervisée            & $\hat{y_i} = \text{sign}({w^{T}x_{i} + b})$                                     & $\frac{1}{2}||w||^2$                                                                               & Standardisation \\
\hline

CSV-C (C-Suport Vector Classification)             & Régression fortement non linéaire avec tolérance aux marges (hyperparamètre non borné)    & Classification supervisée            & $\hat{y_i} = \text{sign}(\sum_{i=1}^{n} \alpha_{i}y_{i}K(x_{i},x_{j}) + b)$     & $\frac{1}{2}||w||^2 + C\sum_{i=1}^{n}\xi_{i}$                                                      & Standardisation \\
\hline

$\nu$-SVM ($\nu$-Suport Vector Machine)            & Régression fortement non linéaire avec tolérance aux marges (hyperparamètre borné)        & Classification supervisée            & $\hat{y_i} = \text{sign}( \sum_{i=1}^{n} \alpha_{i}y_{i}K(x_{i},x_{j}) + b)$    & $\frac{1}{2}||w||^2 - \nu \rho + \frac{1}{n}\sum_{i=1}^{n}\xi_{i}$                                 & Standardisation \\
\hline

SVR / $\epsilon$-SVM (Support Vector Regressor)    & Régression fortement non linéaire                                                         & Régression supervisée                & $\hat{y_i} = \sum_{i=1}^{n}(\alpha_{i} - \alpha_{i}^{*}) K(x_{i},x) + b$        & $\frac{1}{2}||w||^2 + C\sum_{i=1}^{n}(\xi_{i} + \xi_{i}^{*})$                                      & Standardisation \\
\hline

$\nu$-SVR ($\nu$-Support Vector Regressor)         & Régression fortement non linéaire                                                         & Régression supervisée                & $\hat{y_i} = \sum_{i=1}^{n}(\alpha_{i} - \alpha_{i}^{*}) K(x_{i},x) + b$        & $\frac{1}{2}||w||^2 + C(\nu\epsilon + \frac{1}{n}\sum_{i=1}^{n}(\xi_{i} + \xi_{i}^{*}))$           & Standardisation \\
\hline

\end{tabularx}
\caption{Synthèse des modèles de Machine Learning - vecteurs support}
\end{table}

\newpage

\begin{table}[h!]
\scriptsize
\begin{tabularx}{\textwidth}{|C|C|C|C|C|C|C|}
\hline
\textbf{Modèle}                                    & \textbf{Quand l'utiliser}                                                                 & \textbf{Paradigme}                   & \textbf{Formule}                                                                & \textbf{Quantité à minimiser}                                                                      & \textbf{Préparation des données}  \\
\hline

Decision tree                                      & Classification                                                                            & Classification supervisée            & Partitionnement récursif                                                        & $\sum_{i=1}^{n}(y_i - \hat{y_i})²$ dans les feuilles                                               & -               \\
\hline

Random Forest                                      & Classification, robustesse, relations non linéaires                                       & Méthode ensembliste (classification) & Moyenne ou classe majoritaire des arbres                                        & $\sum_{i=1}^{n}(y_i - \hat{y_i})²$ out-of-bag                                                      & -               \\
\hline

Gradient Boosting / XGBoost                        & Classification, robustesse, relations fortement non linéaire                              & Ensemble boosting                    & Actualisation d'un arbre faible                                                 & $-\sum y\log(\hat y)$                                                                              & -               \\
\hline

\end{tabularx}
\caption{Synthèse des modèles de Machine Learning - méthodes d'ensemble}
\end{table}

\newpage

\begin{table}[h!]
\scriptsize
\begin{tabularx}{\textwidth}{|C|C|C|C|C|C|C|}
\hline
\textbf{Modèle}      & \textbf{Quand l'utiliser}                                           & \textbf{Paradigme}                   & \textbf{Formule}                                                                & \textbf{Quantité à minimiser}         & \textbf{Préparation des données}  \\
\hline

AR                   & Séries temporelles                                                  & Modèle autorégressif                 &                                                                                 & -                                     & Stationnarité                      \\
\hline

MA                   & Séries temporelles                                                  & Modèle autorégressif                 &                                                                                 & -                                     & Stationnarité                      \\
\hline

ARMA                 & Séries temporelles                                                  & Modèle autorégressif                 &                                                                                 & -                                     & Stationnarité                      \\
\hline

ARIMA                & Séries temporelles                                                  & Modèle autorégressif                 &                                                                                 & -                                     & Stationnarité                      \\
\hline

SARIMA               & Séries temporelles, saisonnarité                                    & Modèle autorégressif                 &                                                                                 & -                                     & Stationnarité                      \\
\hline

SARIMAX              & Séries temporelles, saisonnarité avec variables exogènes            & Modèle autorégressif                 &                                                                                 & -                                     & Stationnarité                      \\
\hline


\end{tabularx}
\caption{Synthèse des modèles de Machine Learning - processus autorégressifs}
\end{table}

\newpage

\begin{table}[h!]
\scriptsize
\begin{tabularx}{\textwidth}{|C|C|C|C|C|C|C|}
\hline
\textbf{Modèle}      & \textbf{Quand l'utiliser}                                           & \textbf{Paradigme}                   & \textbf{Formule}                                                                & \textbf{Quantité à minimiser}         & \textbf{Préparation des données}  \\
\hline

AR                   & Séries temporelles                                                  & Modèle autorégressif                 &                                                                                 & -                                     & Stationnarité                      \\
\hline


\end{tabularx}
\caption{Synthèse des modèles de Machine Learning - clustering}
\end{table}

\newpage

\begin{table}[h!]
\scriptsize
\begin{tabularx}{\textwidth}{|C|C|C|C|C|C|C|}
\hline
\textbf{Modèle}      & \textbf{Quand l'utiliser}                                           & \textbf{Paradigme}                   & \textbf{Formule}                                                                & \textbf{Quantité à minimiser}         & \textbf{Préparation des données}  \\
\hline

AR                   & Séries temporelles                                                  & Modèle autorégressif                 &                                                                                 & -                                     & Stationnarité                      \\
\hline


\end{tabularx}
\caption{Synthèse des modèles de Machine Learning - réduction de dimension}
\end{table}


\end{document}\documentclass[12pt,french]{report}

% This is the main framework I use for my LateX documents.
% main.tex by Alexis GRACIAS

%%%%%%%%%%%%
% PACKAGES %
%%%%%%%%%%%%

% TABLE OF CONTENT FOR CHAPTERS

\usepackage{minitoc}
\usepackage{minitoc}
\mtcsettitle{minitoc}{Sommaire du chapitre} % Change le titre de la minitoc
\mtcsetfont{minitoc}{section}{\normalsize\bfseries} % Style des sections dans la minitoc
\mtcsetfont{minitoc}{subsection}{\small\itshape} % Style des sous-sections
\mtcsetrules{minitoc}{off} % Désactive les lignes sous les titres de section
\mtcsetoffset{minitoc}{10pt} % Ajoute un espacement autour de la minitoc

\dominitoc % Active les mini-tables des matières
% Personnalisation du titre des mini tables des matières
\renewcommand{\mtctitle}{}


% GEOMETRY

\usepackage[left=2cm,right=2cm,top=2cm,bottom=2cm]{geometry}

% LASTAPGE

\usepackage{lastpage}

% HEADERS AND FOOTERS
\usepackage{fancyhdr}

% LANGUAGE

\usepackage{babel}

% REFERENCES

\usepackage{hyperref} % Add links to the table of content, files and website

% GRAPHS

\usepackage{tikz} % Graph package

% IMAGES

\usepackage{graphicx} % Required for inserting images
\usepackage{tabto}

% COLOR

\usepackage[skins,theorems]{tcolorbox} % Frame color package
\usepackage{amssymb}
\usepackage[dvipsnames]{xcolor} % Text color package and more colors
\usepackage{tcolorbox}

% Custom colors
\definecolor{ao}{rgb}{0.0, 0.0, 1.0}
\definecolor{coolblack}{rgb}{0.0, 0.18, 0.39}
\definecolor{cyan}{rgb}{0.0, 1.0, 1.0}
\definecolor{glaucous}{rgb}{0.38, 0.51, 0.71}
\definecolor{electricultramarine}{rgb}{0.25, 0.0, 1.0}

% CAPTIONS

\usepackage{caption}
\usepackage{subcaption}
\usepackage[utf8]{inputenc}

% STYLE

% Chapter's title style
\usepackage[T1]{fontenc}
\usepackage{titlesec, blindtext, color}
\definecolor{gray75}{gray}{0.75}
\newcommand{\hsp}{\hspace{20pt}}
\titleformat{\chapter}[hang]{\Huge\bfseries}{\thechapter\hsp\textcolor{gray75}{|}\hsp}{0pt}{\Huge\bfseries}

% Table of contents style
\makeatletter
\renewcommand*{\l@chapter}[2]{%         % Chapter label format
  \ifnum \c@tocdepth > \m@ne            % Check: Maximum depth > -1
    \addpenalty{-\@highpenalty}%        % Encourage prior line breaking
    \vskip 1.0em \@plus\p@%             % Prior vertical whitespace
    \setlength\@tempdima{1.5em}%        % Set temporary length
    \begingroup                         % Begin closed group
      \parindent\z@                     % No paragraph indentation
      \rightskip\@tocrmarg              % Reserved right whitespace
      \parfillskip -\@tocrmarg          % Prevent box from moving
      \leavevmode                       % Force horizontal mode
      \advance\leftskip\@tempdima       % Left whitespace
      \hskip -\leftskip                 % First line alignment
      {\bfseries #1}%           % Chapter name and format
      \nobreak                          % Avoid lineskip
      \leaders\hbox{$%                  % Command to replicate dots
        \m@th                           % Math mode for alignment
        \mkern\@dotsep mu               % Whitespace before a dot
        \hbox{.}%                       % Dot separator
        \mkern\@dotsep mu               % Whitespace after a dot
      $}%
      \hfill                            % Flush to the right
      \nobreak\hb@xt@\@pnumwidth{%      % Chapter page’s number box
        \hss #2%                        % Page number with buffer
      }%
      \par                              % Insert line skip
      \penalty\@highpenalty             % Return to default penalty
    \endgroup                           % End closed group
  \fi
}
\makeatother


% MATH

\usepackage{amsmath,amsfonts,mathtools,stmaryrd} % Math libraries

\usepackage{listings} % required for specific languages
\lstset{ % Set listing package options
    language=bash, % choose the language of the code
    basicstyle=\fontfamily{pcr}\selectfont\footnotesize\color{red},
    keywordstyle=\color{black}\bfseries, % style for keywords
    numbers=none, % where to put the line-numbers
    numberstyle=\tiny, % the size of the fonts that are used for the line-numbers     
    backgroundcolor=\color{white},
    showspaces=false, % show spaces adding particular underscores
    showstringspaces=false, % underline spaces within strings
    showtabs=false, % show tabs within strings adding particular underscores
    frame=single, % adds a frame around the code
    tabsize=2, % sets default tabsize to 2 spaces
    rulesepcolor=\color{gray},
    rulecolor=\color{black},
    captionpos=b, % sets the caption-position to bottom
    breaklines=true, % sets automatic line breaking
    breakatwhitespace=false, 
}

% Tabular

\usepackage{tabularx}
\usepackage{ragged2e}

\newcolumntype{L}{>{\RaggedRight\arraybackslash}X} 
\newcolumntype{C}{>{\centering\arraybackslash}X}

% OTHER

\usepackage{pdfpages}

% SCRIPTS (python)

% Load libraries
\usepackage{tcolorbox}
\tcbuselibrary{minted,breakable,xparse,skins}
\usepackage{minted}
\usepackage{xcolor}

\definecolor{bg}{gray}{0.95}
\DeclareTCBListing{mintedbox}{O{}m!O{}}{%
  breakable=true,
  listing engine=minted,
  listing only,
  minted language=#2,
  minted style=default,
  minted options={%
    linenos,
    gobble=0,
    breaklines=true,
    breakafter=,,
    fontsize=\small,
    numbersep=8pt,
    #1},
  boxsep=0pt,
  left skip=0pt,
  right skip=0pt,
  left=25pt,
  right=0pt,
  top=3pt,
  bottom=3pt,
  arc=5pt,
  leftrule=0pt,
  rightrule=0pt,
  bottomrule=2pt,
  toprule=2pt,
  colback=bg,
  colframe=gray!70,
  enhanced,
  overlay={%
    \begin{tcbclipinterior}
    \fill[gray!20!white] (frame.south west) rectangle ([xshift=20pt]frame.north west);
    \end{tcbclipinterior}},
  #3}

%\usepackage{eso-pic,lipsum}
%\AddToShipoutPicture{%
%	\AtTextCenter{%
%		\fboxsep5mm \fboxrule=0.8pt
%		\makebox(0,0)[c]{\fbox{\rule{0pt}\textheight\rule\textwidth{0pt}}}%
%	}%
%}

%%%%%%%%%%%%
% COMMANDS %
%%%%%%%%%%%%

\lstset{aboveskip=\baselineskip,belowskip=\baselineskip,basicstyle=\ttfamily} % Formating line break after bash commands

% Get rid of 0. chapter's number
\makeatletter 
\renewcommand{\thesection}{%
  \ifnum\c@chapter<1 \@arabic\c@section
  \else \thechapter.\@arabic\c@section
  \fi
}
\makeatother

% Rules for \bar{x} and \overline{x} commands
\makeatletter
\newcommand*{\Xbar}{}%
\DeclareRobustCommand*{\Xbar}{%
  \mathpalette\@Xbar{}%
}


% THEOREMS

\definecolor{electricultramarine}{rgb}{0.25, 0.0, 1.0}

\tcbuselibrary{theorems} 
\newtcbtheorem
  []% init options
  {Syntax}% name
  {Syntax}% title
  {%
    colback=BrickRed!2,
    colframe=BrickRed!35!BrickRed,
    fonttitle=\bfseries,
  }% options
  {def}% prefix

\tcbuselibrary{theorems} 
\newtcbtheorem
  []% init options
  {Input}% name
  {Input}% title
  {%
    colback=BlueViolet!5,
    colframe=BlueViolet!35!BlueViolet,
    fonttitle=\bfseries,
  }% options
  {def}% prefix

\tcbuselibrary{theorems} 
\newtcbtheorem
  []% init options
  {Output}% name
  {Output}% title
  {%
    colback=Emerald!5,
    colframe=Emerald!35!Emerald,
    fonttitle=\bfseries,
  }% options
  {def}% prefix

% FORMULAS

\tcbset{highlight math style={enhanced,
  colframe=red,colback=white,arc=0pt,boxrule=1pt}}

% CUSTOM COMMANDS

\newcommand*{\@Xbar}[2]{%
  % #1: math style
  % #2: unused (empty)
  \sbox0{$#1\mathrm{X}\m@th$}%
  \sbox2{$#1X\m@th$}%
  \rlap{%
    \hbox to\wd2{%
      \hfill
      $\overline{%
        \vrule width 0pt height\ht0 %
        \kern\wd0 %
      }$%
    }%
  }%
  \copy2 %
}
\makeatother

%%%%%%%%%%%%%%%%%%%%%%%%%%%%%%
% DECLARE UNICODE CHARACTERS %
%%%%%%%%%%%%%%%%%%%%%%%%%%%%%%

% Main
\DeclareUnicodeCharacter{2500}{\textendash}  % ─
\DeclareUnicodeCharacter{2502}{\textbar}     % │
\DeclareUnicodeCharacter{250C}{\textup{┌}}   % ┌
\DeclareUnicodeCharacter{2510}{\textup{┐}}   % ┐
\DeclareUnicodeCharacter{2514}{\textup{└}}   % └
\DeclareUnicodeCharacter{2518}{\textup{┘}}   % ┘
\DeclareUnicodeCharacter{251C}{\textup{├}}   % ├
\DeclareUnicodeCharacter{2524}{\textup{┤}}   % ┤
\DeclareUnicodeCharacter{252C}{\textup{┬}}   % ┬
\DeclareUnicodeCharacter{2534}{\textup{┴}}   % ┴
\DeclareUnicodeCharacter{253C}{\textup{┼}}   % ┼
\DeclareUnicodeCharacter{2550}{\textup{═}}   % ═
\DeclareUnicodeCharacter{2551}{\textup{║}}   % ║
\DeclareUnicodeCharacter{2552}{\textup{╒}}   % ╒
\DeclareUnicodeCharacter{2553}{\textup{╓}}   % ╓
\DeclareUnicodeCharacter{2554}{\textup{╔}}   % ╔
\DeclareUnicodeCharacter{2555}{\textup{╕}}   % ╕
\DeclareUnicodeCharacter{2556}{\textup{╖}}   % ╖
\DeclareUnicodeCharacter{2557}{\textup{╗}}   % ╗
\DeclareUnicodeCharacter{2558}{\textup{╘}}   % ╘
\DeclareUnicodeCharacter{2559}{\textup{╙}}   % ╙
\DeclareUnicodeCharacter{255A}{\textup{╚}}   % ╚
\DeclareUnicodeCharacter{255B}{\textup{╛}}   % ╛
\DeclareUnicodeCharacter{255C}{\textup{╜}}   % ╜
\DeclareUnicodeCharacter{255D}{\textup{╝}}   % ╝
\DeclareUnicodeCharacter{2560}{\textup{╠}}   % ╠
\DeclareUnicodeCharacter{2561}{\textup{╡}}   % ╡
\DeclareUnicodeCharacter{2562}{\textup{╢}}   % ╢
\DeclareUnicodeCharacter{2563}{\textup{╣}}   % ╣
\DeclareUnicodeCharacter{2564}{\textup{╤}}   % ╤
\DeclareUnicodeCharacter{2565}{\textup{╥}}   % ╥
\DeclareUnicodeCharacter{2566}{\textup{╦}}   % ╦
\DeclareUnicodeCharacter{2567}{\textup{╧}}   % ╧
\DeclareUnicodeCharacter{2568}{\textup{╨}}   % ╨
\DeclareUnicodeCharacter{2569}{\textup{╩}}   % ╩
\DeclareUnicodeCharacter{256A}{\textup{╪}}   % ╪
\DeclareUnicodeCharacter{256B}{\textup{╫}}   % ╫
\DeclareUnicodeCharacter{256C}{\textup{╬}}   % ╬

% Blocks and bars
\DeclareUnicodeCharacter{2580}{\textup{▀}}   % ▀
\DeclareUnicodeCharacter{2584}{\textup{▄}}   % ▄
\DeclareUnicodeCharacter{2588}{\textup{█}}   % █
\DeclareUnicodeCharacter{258C}{\textup{▌}}   % ▌
\DeclareUnicodeCharacter{2590}{\textup{▐}}   % ▐
\DeclareUnicodeCharacter{2591}{\textup{░}}   % ░
\DeclareUnicodeCharacter{2592}{\textup{▒}}   % ▒
\DeclareUnicodeCharacter{2593}{\textup{▓}}   % ▓
\DeclareUnicodeCharacter{25A0}{\textup{■}}   % ■
\DeclareUnicodeCharacter{25A1}{\textup{□}}   % □
\DeclareUnicodeCharacter{25AA}{\textup{▪}}   % ▪
\DeclareUnicodeCharacter{25AB}{\textup{▫}}   % ▫

% Geometry
\DeclareUnicodeCharacter{25B6}{\textup{▶}}   % ▶
\DeclareUnicodeCharacter{25C0}{\textup{◁}}   % ◁
\DeclareUnicodeCharacter{25FB}{\textup{▫}}   % ▫
\DeclareUnicodeCharacter{25FD}{\textup{▽}}   % ▽
\DeclareUnicodeCharacter{25FE}{\textup{▾}}   % ▾

% Currency
\DeclareUnicodeCharacter{20AC}{\textup{€}}   % €
\DeclareUnicodeCharacter{00A3}{\textup{£}}   % £
\DeclareUnicodeCharacter{20B9}{\textup{₹}}   % ₹

% Arrows
\DeclareUnicodeCharacter{2190}{\textup{←}}   % ←
\DeclareUnicodeCharacter{2191}{\textup{↑}}   % ↑
\DeclareUnicodeCharacter{2192}{\textup{→}}   % →
\DeclareUnicodeCharacter{2193}{\textup{↓}}   % ↓
\DeclareUnicodeCharacter{21D0}{\textup{⇐}}   % ⇐
\DeclareUnicodeCharacter{21D1}{\textup{⇑}}   % ⇑
\DeclareUnicodeCharacter{21D2}{\textup{⇒}}   % ⇒
\DeclareUnicodeCharacter{21D3}{\textup{⇓}}   % ⇓

% Miscellaneous
\DeclareUnicodeCharacter{2605}{\textup{★}}   % ★
\DeclareUnicodeCharacter{2606}{\textup{☆}}   % ☆
\DeclareUnicodeCharacter{2610}{\textup{☐}}   % ☐
\DeclareUnicodeCharacter{2611}{\textup{☑}}   % ☑
\DeclareUnicodeCharacter{2612}{\textup{☒}}   % ☒
\DeclareUnicodeCharacter{2620}{\textup{☠}}   % ☠
\DeclareUnicodeCharacter{2622}{\textup{☢}}   % ☢
\DeclareUnicodeCharacter{2623}{\textup{☣}}   % ☣
\DeclareUnicodeCharacter{2626}{\textup{☦}}   % ☦
\DeclareUnicodeCharacter{262A}{\textup{☪}}   % ☪
\DeclareUnicodeCharacter{262F}{\textup{☯}}   % ☯
\DeclareUnicodeCharacter{2660}{\textup{♠}}   % ♠
\DeclareUnicodeCharacter{2665}{\textup{♥}}   % ♥
\DeclareUnicodeCharacter{2666}{\textup{♦}}   % ♦
\DeclareUnicodeCharacter{2663}{\textup{♣}}   % ♣

%%%%%%%%%%%%%%%%%%%%%%
% DOCUMENTS SETTINGS %
%%%%%%%%%%%%%%%%%%%%%%

% Fist page
\newcommand{\HRule}{\rule{\linewidth}{0.5mm}}

\title{
    \HRule \\[0.4cm] \Huge \bfseries 
    Machine learning cheatsheet 
    \\[0.15cm] \HRule \\[0.4cm]}
\author{\Large Alexis GRACIAS}


% Document
\begin{document}


\maketitle

\dominitoc % Initialization
\large \tableofcontents
\normalsize % Text size

% Headers and footers
\renewcommand{\chaptermark}[1]{ \markboth{#1}{} }
\pagestyle{fancy}
%... then configure it.
\fancyhead{} % clear all header fields
\fancyhead[RO,LE]{\leftmark}
\fancyhead[LO,LE]{A. GRACIAS}
\fancyfoot{} % clear all footer fields

% Page number
\makeatletter
\renewcommand{\@evenfoot}{\makebox[\textwidth][c]{page {\thepage} sur \pageref{LastPage}}}
\renewcommand{\@oddfoot}{\@evenfoot}
\makeatother

% ITEMIZE (lists)
\renewcommand{\labelitemi}{$\bullet$}
\renewcommand{\labelitemii}{---}
\renewcommand{\labelitemiii}{-}
\renewcommand{\labelitemiv}{.}

%%%%%%%%%%%
% INCLUDE %
%%%%%%%%%%%

\chapter{Formulation mathématique d'un \textit{dataset}}


\newpage

\chapter{Métriques}
\section{Régression}

\begin{itemize}

    \item \textbf{Mean Absolute Error} : 
        \begin{equation}
            MSE(Y) = \frac{1}{n}\sum_{i = 1}^{n}|y_{i} - \hat{y_{i}}|
        \end{equation} \\
    \item \textbf{Root Mean Squarred Error} : 
        \begin{equation}
            MSE(Y) = \frac{1}{n}\sum_{i = 1}^{n}(y_{i} - \hat{y_{i}})^{2}
        \end{equation} \\

    \item 
\end{itemize}

\section{Classification}

\begin{itemize}

    \item \textbf{Accuracy score} : 
        \begin{equation}
            \text{Accuracy} = \frac{VP + VN}{VP + VN + FN + FP} = 1 - exactitude
        \end{equation} 

    Avec : \newline

    \begin{itemize}
        \item  VP : Vrai négatif   \\
        \item  FP : Faux positifs  \\
        \item  FN : Faux négatifs  \\
        \item  VP : Vrais positifs \\
    \end{itemize} 

    \item \textbf{Matrice de confusion} \newline
    Une matrice de confusion recense le nombre de faux positifs, faux négatifs, vrai positifs et vrai négatifs 
    sous forme de tableau. Les éléments de cette matrice $2 \times 2$ veulent dire ceci : \newline

    \begin{itemize}
        \item  Vrai négatif (Réel 0, Prédit 0)   \\
        \item  Faux positifs (Réel 0, Prédit 1)  \\
        \item  Faux négatifs (Réel 1, Prédit 0)  \\ 
        \item  Vrais positifs (Réel 1, Prédit 1) \\
    \end{itemize}

    La prédiction est la valeur que le modèle prédit tandis que réel est la valeur réelle, qui peut être observée à *posteriori*
    Par exemple, une matrice de confusion peut ressembler à ça : \newline


    \begin{table}[h!]
    \scriptsize
    \begin{tabularx}{\textwidth}{|C|C|C|}
    \hline
      &\textbf{Prédit : 0}   & \textbf{Prédit : 1}  \\
    \hline
    Réel : 0 & 87 & 13 \\
    \hline
    Réel : 1 & 18 & 61 \\
    \hline
    \end{tabularx}
    \caption{Exemple de matrice de confusion}
    \end{table}

    Ici, le modèle a prédit que 87 + 18 = 105 personnes n'ont pas survécues à l'accident sur les 87 + 13 = 100 personnes qui n'ont réellement pas survécues.
    De même, le modèle prédit que 13 + 61 = 74 personnes on survécues sur 10 + 61 = 71 qui ont réelement survécues. \\

\end{itemize}

\newpage

\chapter{Synthèse des modèles et leurs usages}

\begin{table}[h!]
\scriptsize
\begin{tabularx}{\textwidth}{|C|C|C|C|C|C|C|}
\hline
\textbf{Modèle}                                    & \textbf{Quand l'utiliser}                                                                 & \textbf{Paradigme}                   & \textbf{Formule}                                                                & \textbf{Quantité à minimiser}                                                                      & \textbf{Préparation des données}  \\
\hline

Régression linéaire                                & Relation linéaire entre $X_i$ et $X_j$                                                    & Régression statistique               & $\hat{y} = X\beta + \varepsilon$                                                & $||y - X\beta||^2$                                                                                 & Standardisation \\
\hline

Régression Ridge                                   & Régression avec pénalisation $L_2$ (sélection de variables)                               & Régression statistique               & $\hat{y} = X\beta + \varepsilon$                                                & $||y-X\beta||^2 + \lambda ||\beta||^2$                                                             & Standardisation \\
\hline

Régression Lasso                                   & Régression avec pénalisation $L_1$ (sélection de variables avec annulation)               & Régression statistique               & $\hat{y} = X\beta + \varepsilon$                                                & $||y-X\beta||^2 + \lambda ||\beta||_1$                                                             & Standardisation \\
\hline

Elastic Net                                        & Régression Ridge + Lasso                                                                  & Régression statistique               & $\hat{y} = X\beta + \varepsilon$                                                & $||y-X\beta||^2 + \lambda(\alpha ||\beta||_1 +(1-\alpha)||\beta||_2^2)$                            & Standardisation \\
\hline

Régression logistique                              & Classification binaire                                                                    & Classification binaire               & $\hat{y} = \sigma(X\beta)$                                                      & $-\sum y\log(\hat y)$                                                                              & Standardisation \\
\hline

Régression logistique multinomiale                 & Classification                                                                            & Classification                       &                                                                                 &                                                                                                    & Standardisation \\
\hline

Régression polynomiale                             & Régression non linéaire                                                                   & Régression statistique               & $\hat{y} = \sum_{k=0}^{d} \beta_k x^k$                                          & $\sum_{i=1}^{n}(y_i - \hat{y_i})^2$                                                                & Standardisation \\
\hline

Régression splines                                 & Régression non linéaire, robuste, locale                                                  & Régression statistique               &                                                                                 &                                                                                                    & -               \\
\hline

Régression additive généralisée (GAM)              & Régression fortement non linéaire, robuste, locale                                        & Régression statistique               &                                                                                 &                                                                                                    & -               \\
\hline

Régression isotonique                              & Régression non linéaire monotone                                                          & Régression statistique               &                                                                                 &                                                                                                    & Standardisation \\
\hline

Régression Généralisée (GLM)                       & Régression non linéaire                                                                   & Régression statistique               &                                                                                 &                                                                                                    & Standardisation \\
\hline

Quantile regression                                & Régression fortement non linéaire, robuste, présence d'hétéroscédasticité                 & Régression statistique               &                                                                                 & pinball loss                                                                                       & Standardisation \\
\hline

Processus gaussiens (GP)                           & Régression fortement non linéaire                                                         & Régression statistique               &                                                                                 &                                                                                                    & - \\
\hline

Huber regression                                   & Régression fortement non linéaire, robuste, insensible aux outliers                       & Régression statistique               &                                                                                 & huber loss                                                                                         & - \\
\hline

Régression PLS                                     & Régression non linéaire avec analyse des composantes principales                          & Régression non supervisée            &                                                                                 &                                                                                                    & Standardisation \\
\hline

KNN regression                                     & Régression locale / ponctuelle avec beaucoup de données                                   & Régression statistique               & $\hat{y} = \frac{1}{k}\sum_{i \in \mathcal{N}_k(x)} y_i$                        & $\sum_{i=1}^{n}(y_i - \hat{y_i})^2$                                                                & Standardisation \\
\hline

\end{tabularx}
\caption{Synthèse des modèles de Machine Learning - régression}
\end{table}

\newpage

\begin{table}[h!]
\scriptsize
\begin{tabularx}{\textwidth}{|C|C|C|C|C|C|C|}
\hline
\textbf{Modèle}                                    & \textbf{Quand l'utiliser}                                                                 & \textbf{Paradigme}                   & \textbf{Formule}                                                                & \textbf{Quantité à minimiser}                                                                      & \textbf{Préparation des données}  \\
\hline

SVM (Support Vector Machine)                       & Classification fortement non linéaire à marge dure                                        & Classification supervisée            & $\hat{y_i} = \text{sign}({w^{T}x_{i} + b})$                                     & $\frac{1}{2}||w||^2$                                                                               & Standardisation \\
\hline

CSV-C (C-Suport Vector Classification)             & Régression fortement non linéaire avec tolérance aux marges (hyperparamètre non borné)    & Classification supervisée            & $\hat{y_i} = \text{sign}(\sum_{i=1}^{n} \alpha_{i}y_{i}K(x_{i},x_{j}) + b)$     & $\frac{1}{2}||w||^2 + C\sum_{i=1}^{n}\xi_{i}$                                                      & Standardisation \\
\hline

$\nu$-SVM ($\nu$-Suport Vector Machine)            & Régression fortement non linéaire avec tolérance aux marges (hyperparamètre borné)        & Classification supervisée            & $\hat{y_i} = \text{sign}( \sum_{i=1}^{n} \alpha_{i}y_{i}K(x_{i},x_{j}) + b)$    & $\frac{1}{2}||w||^2 - \nu \rho + \frac{1}{n}\sum_{i=1}^{n}\xi_{i}$                                 & Standardisation \\
\hline

SVR / $\epsilon$-SVM (Support Vector Regressor)    & Régression fortement non linéaire                                                         & Régression supervisée                & $\hat{y_i} = \sum_{i=1}^{n}(\alpha_{i} - \alpha_{i}^{*}) K(x_{i},x) + b$        & $\frac{1}{2}||w||^2 + C\sum_{i=1}^{n}(\xi_{i} + \xi_{i}^{*})$                                      & Standardisation \\
\hline

$\nu$-SVR ($\nu$-Support Vector Regressor)         & Régression fortement non linéaire                                                         & Régression supervisée                & $\hat{y_i} = \sum_{i=1}^{n}(\alpha_{i} - \alpha_{i}^{*}) K(x_{i},x) + b$        & $\frac{1}{2}||w||^2 + C(\nu\epsilon + \frac{1}{n}\sum_{i=1}^{n}(\xi_{i} + \xi_{i}^{*}))$           & Standardisation \\
\hline

\end{tabularx}
\caption{Synthèse des modèles de Machine Learning - vecteurs support}
\end{table}

\newpage

\begin{table}[h!]
\scriptsize
\begin{tabularx}{\textwidth}{|C|C|C|C|C|C|C|}
\hline
\textbf{Modèle}                                    & \textbf{Quand l'utiliser}                                                                 & \textbf{Paradigme}                   & \textbf{Formule}                                                                & \textbf{Quantité à minimiser}                                                                      & \textbf{Préparation des données}  \\
\hline

Decision tree                                      & Classification                                                                            & Classification supervisée            & Partitionnement récursif                                                        & $\sum_{i=1}^{n}(y_i - \hat{y_i})²$ dans les feuilles                                               & -               \\
\hline

Random Forest                                      & Classification, robustesse, relations non linéaires                                       & Méthode ensembliste (classification) & Moyenne ou classe majoritaire des arbres                                        & $\sum_{i=1}^{n}(y_i - \hat{y_i})²$ out-of-bag                                                      & -               \\
\hline

Gradient Boosting / XGBoost                        & Classification, robustesse, relations fortement non linéaire                              & Ensemble boosting                    & Actualisation d'un arbre faible                                                 & $-\sum y\log(\hat y)$                                                                              & -               \\
\hline

\end{tabularx}
\caption{Synthèse des modèles de Machine Learning - méthodes d'ensemble}
\end{table}

\newpage

\begin{table}[h!]
\scriptsize
\begin{tabularx}{\textwidth}{|C|C|C|C|C|C|C|}
\hline
\textbf{Modèle}      & \textbf{Quand l'utiliser}                                           & \textbf{Paradigme}                   & \textbf{Formule}                                                                & \textbf{Quantité à minimiser}         & \textbf{Préparation des données}  \\
\hline

AR                   & Séries temporelles                                                  & Modèle autorégressif                 &                                                                                 & -                                     & Stationnarité                      \\
\hline

MA                   & Séries temporelles                                                  & Modèle autorégressif                 &                                                                                 & -                                     & Stationnarité                      \\
\hline

ARMA                 & Séries temporelles                                                  & Modèle autorégressif                 &                                                                                 & -                                     & Stationnarité                      \\
\hline

ARIMA                & Séries temporelles                                                  & Modèle autorégressif                 &                                                                                 & -                                     & Stationnarité                      \\
\hline

SARIMA               & Séries temporelles, saisonnarité                                    & Modèle autorégressif                 &                                                                                 & -                                     & Stationnarité                      \\
\hline

SARIMAX              & Séries temporelles, saisonnarité avec variables exogènes            & Modèle autorégressif                 &                                                                                 & -                                     & Stationnarité                      \\
\hline


\end{tabularx}
\caption{Synthèse des modèles de Machine Learning - processus autorégressifs}
\end{table}

\newpage

\begin{table}[h!]
\scriptsize
\begin{tabularx}{\textwidth}{|C|C|C|C|C|C|C|}
\hline
\textbf{Modèle}      & \textbf{Quand l'utiliser}                                           & \textbf{Paradigme}                   & \textbf{Formule}                                                                & \textbf{Quantité à minimiser}         & \textbf{Préparation des données}  \\
\hline

AR                   & Séries temporelles                                                  & Modèle autorégressif                 &                                                                                 & -                                     & Stationnarité                      \\
\hline


\end{tabularx}
\caption{Synthèse des modèles de Machine Learning - clustering}
\end{table}

\newpage

\begin{table}[h!]
\scriptsize
\begin{tabularx}{\textwidth}{|C|C|C|C|C|C|C|}
\hline
\textbf{Modèle}      & \textbf{Quand l'utiliser}                                           & \textbf{Paradigme}                   & \textbf{Formule}                                                                & \textbf{Quantité à minimiser}         & \textbf{Préparation des données}  \\
\hline

AR                   & Séries temporelles                                                  & Modèle autorégressif                 &                                                                                 & -                                     & Stationnarité                      \\
\hline


\end{tabularx}
\caption{Synthèse des modèles de Machine Learning - réduction de dimension}
\end{table}


\end{document}\documentclass[12pt,french]{report}

% This is the main framework I use for my LateX documents.
% main.tex by Alexis GRACIAS

%%%%%%%%%%%%
% PACKAGES %
%%%%%%%%%%%%

% TABLE OF CONTENT FOR CHAPTERS

\usepackage{minitoc}
\usepackage{minitoc}
\mtcsettitle{minitoc}{Sommaire du chapitre} % Change le titre de la minitoc
\mtcsetfont{minitoc}{section}{\normalsize\bfseries} % Style des sections dans la minitoc
\mtcsetfont{minitoc}{subsection}{\small\itshape} % Style des sous-sections
\mtcsetrules{minitoc}{off} % Désactive les lignes sous les titres de section
\mtcsetoffset{minitoc}{10pt} % Ajoute un espacement autour de la minitoc

\dominitoc % Active les mini-tables des matières
% Personnalisation du titre des mini tables des matières
\renewcommand{\mtctitle}{}


% GEOMETRY

\usepackage[left=2cm,right=2cm,top=2cm,bottom=2cm]{geometry}

% LASTAPGE

\usepackage{lastpage}

% HEADERS AND FOOTERS
\usepackage{fancyhdr}

% LANGUAGE

\usepackage{babel}

% REFERENCES

\usepackage{hyperref} % Add links to the table of content, files and website

% GRAPHS

\usepackage{tikz} % Graph package

% IMAGES

\usepackage{graphicx} % Required for inserting images
\usepackage{tabto}

% COLOR

\usepackage[skins,theorems]{tcolorbox} % Frame color package
\usepackage{amssymb}
\usepackage[dvipsnames]{xcolor} % Text color package and more colors
\usepackage{tcolorbox}

% Custom colors
\definecolor{ao}{rgb}{0.0, 0.0, 1.0}
\definecolor{coolblack}{rgb}{0.0, 0.18, 0.39}
\definecolor{cyan}{rgb}{0.0, 1.0, 1.0}
\definecolor{glaucous}{rgb}{0.38, 0.51, 0.71}
\definecolor{electricultramarine}{rgb}{0.25, 0.0, 1.0}

% CAPTIONS

\usepackage{caption}
\usepackage{subcaption}
\usepackage[utf8]{inputenc}

% STYLE

% Chapter's title style
\usepackage[T1]{fontenc}
\usepackage{titlesec, blindtext, color}
\definecolor{gray75}{gray}{0.75}
\newcommand{\hsp}{\hspace{20pt}}
\titleformat{\chapter}[hang]{\Huge\bfseries}{\thechapter\hsp\textcolor{gray75}{|}\hsp}{0pt}{\Huge\bfseries}

% Table of contents style
\makeatletter
\renewcommand*{\l@chapter}[2]{%         % Chapter label format
  \ifnum \c@tocdepth > \m@ne            % Check: Maximum depth > -1
    \addpenalty{-\@highpenalty}%        % Encourage prior line breaking
    \vskip 1.0em \@plus\p@%             % Prior vertical whitespace
    \setlength\@tempdima{1.5em}%        % Set temporary length
    \begingroup                         % Begin closed group
      \parindent\z@                     % No paragraph indentation
      \rightskip\@tocrmarg              % Reserved right whitespace
      \parfillskip -\@tocrmarg          % Prevent box from moving
      \leavevmode                       % Force horizontal mode
      \advance\leftskip\@tempdima       % Left whitespace
      \hskip -\leftskip                 % First line alignment
      {\bfseries #1}%           % Chapter name and format
      \nobreak                          % Avoid lineskip
      \leaders\hbox{$%                  % Command to replicate dots
        \m@th                           % Math mode for alignment
        \mkern\@dotsep mu               % Whitespace before a dot
        \hbox{.}%                       % Dot separator
        \mkern\@dotsep mu               % Whitespace after a dot
      $}%
      \hfill                            % Flush to the right
      \nobreak\hb@xt@\@pnumwidth{%      % Chapter page’s number box
        \hss #2%                        % Page number with buffer
      }%
      \par                              % Insert line skip
      \penalty\@highpenalty             % Return to default penalty
    \endgroup                           % End closed group
  \fi
}
\makeatother


% MATH

\usepackage{amsmath,amsfonts,mathtools,stmaryrd} % Math libraries

\usepackage{listings} % required for specific languages
\lstset{ % Set listing package options
    language=bash, % choose the language of the code
    basicstyle=\fontfamily{pcr}\selectfont\footnotesize\color{red},
    keywordstyle=\color{black}\bfseries, % style for keywords
    numbers=none, % where to put the line-numbers
    numberstyle=\tiny, % the size of the fonts that are used for the line-numbers     
    backgroundcolor=\color{white},
    showspaces=false, % show spaces adding particular underscores
    showstringspaces=false, % underline spaces within strings
    showtabs=false, % show tabs within strings adding particular underscores
    frame=single, % adds a frame around the code
    tabsize=2, % sets default tabsize to 2 spaces
    rulesepcolor=\color{gray},
    rulecolor=\color{black},
    captionpos=b, % sets the caption-position to bottom
    breaklines=true, % sets automatic line breaking
    breakatwhitespace=false, 
}

% Tabular

\usepackage{tabularx}
\usepackage{ragged2e}

\newcolumntype{L}{>{\RaggedRight\arraybackslash}X} 
\newcolumntype{C}{>{\centering\arraybackslash}X}

% OTHER

\usepackage{pdfpages}

% SCRIPTS (python)

% Load libraries
\usepackage{tcolorbox}
\tcbuselibrary{minted,breakable,xparse,skins}
\usepackage{minted}
\usepackage{xcolor}

\definecolor{bg}{gray}{0.95}
\DeclareTCBListing{mintedbox}{O{}m!O{}}{%
  breakable=true,
  listing engine=minted,
  listing only,
  minted language=#2,
  minted style=default,
  minted options={%
    linenos,
    gobble=0,
    breaklines=true,
    breakafter=,,
    fontsize=\small,
    numbersep=8pt,
    #1},
  boxsep=0pt,
  left skip=0pt,
  right skip=0pt,
  left=25pt,
  right=0pt,
  top=3pt,
  bottom=3pt,
  arc=5pt,
  leftrule=0pt,
  rightrule=0pt,
  bottomrule=2pt,
  toprule=2pt,
  colback=bg,
  colframe=gray!70,
  enhanced,
  overlay={%
    \begin{tcbclipinterior}
    \fill[gray!20!white] (frame.south west) rectangle ([xshift=20pt]frame.north west);
    \end{tcbclipinterior}},
  #3}

%\usepackage{eso-pic,lipsum}
%\AddToShipoutPicture{%
%	\AtTextCenter{%
%		\fboxsep5mm \fboxrule=0.8pt
%		\makebox(0,0)[c]{\fbox{\rule{0pt}\textheight\rule\textwidth{0pt}}}%
%	}%
%}

%%%%%%%%%%%%
% COMMANDS %
%%%%%%%%%%%%

\lstset{aboveskip=\baselineskip,belowskip=\baselineskip,basicstyle=\ttfamily} % Formating line break after bash commands

% Get rid of 0. chapter's number
\makeatletter 
\renewcommand{\thesection}{%
  \ifnum\c@chapter<1 \@arabic\c@section
  \else \thechapter.\@arabic\c@section
  \fi
}
\makeatother

% Rules for \bar{x} and \overline{x} commands
\makeatletter
\newcommand*{\Xbar}{}%
\DeclareRobustCommand*{\Xbar}{%
  \mathpalette\@Xbar{}%
}


% THEOREMS

\definecolor{electricultramarine}{rgb}{0.25, 0.0, 1.0}

\tcbuselibrary{theorems} 
\newtcbtheorem
  []% init options
  {Syntax}% name
  {Syntax}% title
  {%
    colback=BrickRed!2,
    colframe=BrickRed!35!BrickRed,
    fonttitle=\bfseries,
  }% options
  {def}% prefix

\tcbuselibrary{theorems} 
\newtcbtheorem
  []% init options
  {Input}% name
  {Input}% title
  {%
    colback=BlueViolet!5,
    colframe=BlueViolet!35!BlueViolet,
    fonttitle=\bfseries,
  }% options
  {def}% prefix

\tcbuselibrary{theorems} 
\newtcbtheorem
  []% init options
  {Output}% name
  {Output}% title
  {%
    colback=Emerald!5,
    colframe=Emerald!35!Emerald,
    fonttitle=\bfseries,
  }% options
  {def}% prefix

% FORMULAS

\tcbset{highlight math style={enhanced,
  colframe=red,colback=white,arc=0pt,boxrule=1pt}}

% CUSTOM COMMANDS

\newcommand*{\@Xbar}[2]{%
  % #1: math style
  % #2: unused (empty)
  \sbox0{$#1\mathrm{X}\m@th$}%
  \sbox2{$#1X\m@th$}%
  \rlap{%
    \hbox to\wd2{%
      \hfill
      $\overline{%
        \vrule width 0pt height\ht0 %
        \kern\wd0 %
      }$%
    }%
  }%
  \copy2 %
}
\makeatother

%%%%%%%%%%%%%%%%%%%%%%%%%%%%%%
% DECLARE UNICODE CHARACTERS %
%%%%%%%%%%%%%%%%%%%%%%%%%%%%%%

% Main
\DeclareUnicodeCharacter{2500}{\textendash}  % ─
\DeclareUnicodeCharacter{2502}{\textbar}     % │
\DeclareUnicodeCharacter{250C}{\textup{┌}}   % ┌
\DeclareUnicodeCharacter{2510}{\textup{┐}}   % ┐
\DeclareUnicodeCharacter{2514}{\textup{└}}   % └
\DeclareUnicodeCharacter{2518}{\textup{┘}}   % ┘
\DeclareUnicodeCharacter{251C}{\textup{├}}   % ├
\DeclareUnicodeCharacter{2524}{\textup{┤}}   % ┤
\DeclareUnicodeCharacter{252C}{\textup{┬}}   % ┬
\DeclareUnicodeCharacter{2534}{\textup{┴}}   % ┴
\DeclareUnicodeCharacter{253C}{\textup{┼}}   % ┼
\DeclareUnicodeCharacter{2550}{\textup{═}}   % ═
\DeclareUnicodeCharacter{2551}{\textup{║}}   % ║
\DeclareUnicodeCharacter{2552}{\textup{╒}}   % ╒
\DeclareUnicodeCharacter{2553}{\textup{╓}}   % ╓
\DeclareUnicodeCharacter{2554}{\textup{╔}}   % ╔
\DeclareUnicodeCharacter{2555}{\textup{╕}}   % ╕
\DeclareUnicodeCharacter{2556}{\textup{╖}}   % ╖
\DeclareUnicodeCharacter{2557}{\textup{╗}}   % ╗
\DeclareUnicodeCharacter{2558}{\textup{╘}}   % ╘
\DeclareUnicodeCharacter{2559}{\textup{╙}}   % ╙
\DeclareUnicodeCharacter{255A}{\textup{╚}}   % ╚
\DeclareUnicodeCharacter{255B}{\textup{╛}}   % ╛
\DeclareUnicodeCharacter{255C}{\textup{╜}}   % ╜
\DeclareUnicodeCharacter{255D}{\textup{╝}}   % ╝
\DeclareUnicodeCharacter{2560}{\textup{╠}}   % ╠
\DeclareUnicodeCharacter{2561}{\textup{╡}}   % ╡
\DeclareUnicodeCharacter{2562}{\textup{╢}}   % ╢
\DeclareUnicodeCharacter{2563}{\textup{╣}}   % ╣
\DeclareUnicodeCharacter{2564}{\textup{╤}}   % ╤
\DeclareUnicodeCharacter{2565}{\textup{╥}}   % ╥
\DeclareUnicodeCharacter{2566}{\textup{╦}}   % ╦
\DeclareUnicodeCharacter{2567}{\textup{╧}}   % ╧
\DeclareUnicodeCharacter{2568}{\textup{╨}}   % ╨
\DeclareUnicodeCharacter{2569}{\textup{╩}}   % ╩
\DeclareUnicodeCharacter{256A}{\textup{╪}}   % ╪
\DeclareUnicodeCharacter{256B}{\textup{╫}}   % ╫
\DeclareUnicodeCharacter{256C}{\textup{╬}}   % ╬

% Blocks and bars
\DeclareUnicodeCharacter{2580}{\textup{▀}}   % ▀
\DeclareUnicodeCharacter{2584}{\textup{▄}}   % ▄
\DeclareUnicodeCharacter{2588}{\textup{█}}   % █
\DeclareUnicodeCharacter{258C}{\textup{▌}}   % ▌
\DeclareUnicodeCharacter{2590}{\textup{▐}}   % ▐
\DeclareUnicodeCharacter{2591}{\textup{░}}   % ░
\DeclareUnicodeCharacter{2592}{\textup{▒}}   % ▒
\DeclareUnicodeCharacter{2593}{\textup{▓}}   % ▓
\DeclareUnicodeCharacter{25A0}{\textup{■}}   % ■
\DeclareUnicodeCharacter{25A1}{\textup{□}}   % □
\DeclareUnicodeCharacter{25AA}{\textup{▪}}   % ▪
\DeclareUnicodeCharacter{25AB}{\textup{▫}}   % ▫

% Geometry
\DeclareUnicodeCharacter{25B6}{\textup{▶}}   % ▶
\DeclareUnicodeCharacter{25C0}{\textup{◁}}   % ◁
\DeclareUnicodeCharacter{25FB}{\textup{▫}}   % ▫
\DeclareUnicodeCharacter{25FD}{\textup{▽}}   % ▽
\DeclareUnicodeCharacter{25FE}{\textup{▾}}   % ▾

% Currency
\DeclareUnicodeCharacter{20AC}{\textup{€}}   % €
\DeclareUnicodeCharacter{00A3}{\textup{£}}   % £
\DeclareUnicodeCharacter{20B9}{\textup{₹}}   % ₹

% Arrows
\DeclareUnicodeCharacter{2190}{\textup{←}}   % ←
\DeclareUnicodeCharacter{2191}{\textup{↑}}   % ↑
\DeclareUnicodeCharacter{2192}{\textup{→}}   % →
\DeclareUnicodeCharacter{2193}{\textup{↓}}   % ↓
\DeclareUnicodeCharacter{21D0}{\textup{⇐}}   % ⇐
\DeclareUnicodeCharacter{21D1}{\textup{⇑}}   % ⇑
\DeclareUnicodeCharacter{21D2}{\textup{⇒}}   % ⇒
\DeclareUnicodeCharacter{21D3}{\textup{⇓}}   % ⇓

% Miscellaneous
\DeclareUnicodeCharacter{2605}{\textup{★}}   % ★
\DeclareUnicodeCharacter{2606}{\textup{☆}}   % ☆
\DeclareUnicodeCharacter{2610}{\textup{☐}}   % ☐
\DeclareUnicodeCharacter{2611}{\textup{☑}}   % ☑
\DeclareUnicodeCharacter{2612}{\textup{☒}}   % ☒
\DeclareUnicodeCharacter{2620}{\textup{☠}}   % ☠
\DeclareUnicodeCharacter{2622}{\textup{☢}}   % ☢
\DeclareUnicodeCharacter{2623}{\textup{☣}}   % ☣
\DeclareUnicodeCharacter{2626}{\textup{☦}}   % ☦
\DeclareUnicodeCharacter{262A}{\textup{☪}}   % ☪
\DeclareUnicodeCharacter{262F}{\textup{☯}}   % ☯
\DeclareUnicodeCharacter{2660}{\textup{♠}}   % ♠
\DeclareUnicodeCharacter{2665}{\textup{♥}}   % ♥
\DeclareUnicodeCharacter{2666}{\textup{♦}}   % ♦
\DeclareUnicodeCharacter{2663}{\textup{♣}}   % ♣

%%%%%%%%%%%%%%%%%%%%%%
% DOCUMENTS SETTINGS %
%%%%%%%%%%%%%%%%%%%%%%

% Fist page
\newcommand{\HRule}{\rule{\linewidth}{0.5mm}}

\title{
    \HRule \\[0.4cm] \Huge \bfseries 
    Machine learning cheatsheet 
    \\[0.15cm] \HRule \\[0.4cm]}
\author{\Large Alexis GRACIAS}


% Document
\begin{document}


\maketitle

\dominitoc % Initialization
\large \tableofcontents
\normalsize % Text size

% Headers and footers
\renewcommand{\chaptermark}[1]{ \markboth{#1}{} }
\pagestyle{fancy}
%... then configure it.
\fancyhead{} % clear all header fields
\fancyhead[RO,LE]{\leftmark}
\fancyhead[LO,LE]{A. GRACIAS}
\fancyfoot{} % clear all footer fields

% Page number
\makeatletter
\renewcommand{\@evenfoot}{\makebox[\textwidth][c]{page {\thepage} sur \pageref{LastPage}}}
\renewcommand{\@oddfoot}{\@evenfoot}
\makeatother

% ITEMIZE (lists)
\renewcommand{\labelitemi}{$\bullet$}
\renewcommand{\labelitemii}{---}
\renewcommand{\labelitemiii}{-}
\renewcommand{\labelitemiv}{.}

%%%%%%%%%%%
% INCLUDE %
%%%%%%%%%%%

\chapter{Formulation mathématique d'un \textit{dataset}}


\newpage

\chapter{Métriques}
\section{Régression}

\begin{itemize}

    \item \textbf{Mean Absolute Error} : 
        \begin{equation}
            MSE(Y) = \frac{1}{n}\sum_{i = 1}^{n}|y_{i} - \hat{y_{i}}|
        \end{equation} \\
    \item \textbf{Root Mean Squarred Error} : 
        \begin{equation}
            MSE(Y) = \frac{1}{n}\sum_{i = 1}^{n}(y_{i} - \hat{y_{i}})^{2}
        \end{equation} \\

    \item 
\end{itemize}

\section{Classification}

\begin{itemize}

    \item \textbf{Accuracy score} : 
        \begin{equation}
            \text{Accuracy} = \frac{VP + VN}{VP + VN + FN + FP} = 1 - exactitude
        \end{equation} 

    Avec : \newline

    \begin{itemize}
        \item  VP : Vrai négatif   \\
        \item  FP : Faux positifs  \\
        \item  FN : Faux négatifs  \\
        \item  VP : Vrais positifs \\
    \end{itemize} 

    \item \textbf{Matrice de confusion} \newline
    Une matrice de confusion recense le nombre de faux positifs, faux négatifs, vrai positifs et vrai négatifs 
    sous forme de tableau. Les éléments de cette matrice $2 \times 2$ veulent dire ceci : \newline

    \begin{itemize}
        \item  Vrai négatif (Réel 0, Prédit 0)   \\
        \item  Faux positifs (Réel 0, Prédit 1)  \\
        \item  Faux négatifs (Réel 1, Prédit 0)  \\ 
        \item  Vrais positifs (Réel 1, Prédit 1) \\
    \end{itemize}

    La prédiction est la valeur que le modèle prédit tandis que réel est la valeur réelle, qui peut être observée à *posteriori*
    Par exemple, une matrice de confusion peut ressembler à ça : \newline


    \begin{table}[h!]
    \scriptsize
    \begin{tabularx}{\textwidth}{|C|C|C|}
    \hline
      &\textbf{Prédit : 0}   & \textbf{Prédit : 1}  \\
    \hline
    Réel : 0 & 87 & 13 \\
    \hline
    Réel : 1 & 18 & 61 \\
    \hline
    \end{tabularx}
    \caption{Exemple de matrice de confusion}
    \end{table}

    Ici, le modèle a prédit que 87 + 18 = 105 personnes n'ont pas survécues à l'accident sur les 87 + 13 = 100 personnes qui n'ont réellement pas survécues.
    De même, le modèle prédit que 13 + 61 = 74 personnes on survécues sur 10 + 61 = 71 qui ont réelement survécues. \\

\end{itemize}

\newpage

\chapter{Synthèse des modèles et leurs usages}

\begin{table}[h!]
\scriptsize
\begin{tabularx}{\textwidth}{|C|C|C|C|C|C|C|}
\hline
\textbf{Modèle}                                    & \textbf{Quand l'utiliser}                                                                 & \textbf{Paradigme}                   & \textbf{Formule}                                                                & \textbf{Quantité à minimiser}                                                                      & \textbf{Préparation des données}  \\
\hline

Régression linéaire                                & Relation linéaire entre $X_i$ et $X_j$                                                    & Régression statistique               & $\hat{y} = X\beta + \varepsilon$                                                & $||y - X\beta||^2$                                                                                 & Standardisation \\
\hline

Régression Ridge                                   & Régression avec pénalisation $L_2$ (sélection de variables)                               & Régression statistique               & $\hat{y} = X\beta + \varepsilon$                                                & $||y-X\beta||^2 + \lambda ||\beta||^2$                                                             & Standardisation \\
\hline

Régression Lasso                                   & Régression avec pénalisation $L_1$ (sélection de variables avec annulation)               & Régression statistique               & $\hat{y} = X\beta + \varepsilon$                                                & $||y-X\beta||^2 + \lambda ||\beta||_1$                                                             & Standardisation \\
\hline

Elastic Net                                        & Régression Ridge + Lasso                                                                  & Régression statistique               & $\hat{y} = X\beta + \varepsilon$                                                & $||y-X\beta||^2 + \lambda(\alpha ||\beta||_1 +(1-\alpha)||\beta||_2^2)$                            & Standardisation \\
\hline

Régression logistique                              & Classification binaire                                                                    & Classification binaire               & $\hat{y} = \sigma(X\beta)$                                                      & $-\sum y\log(\hat y)$                                                                              & Standardisation \\
\hline

Régression logistique multinomiale                 & Classification                                                                            & Classification                       &                                                                                 &                                                                                                    & Standardisation \\
\hline

Régression polynomiale                             & Régression non linéaire                                                                   & Régression statistique               & $\hat{y} = \sum_{k=0}^{d} \beta_k x^k$                                          & $\sum_{i=1}^{n}(y_i - \hat{y_i})^2$                                                                & Standardisation \\
\hline

Régression splines                                 & Régression non linéaire, robuste, locale                                                  & Régression statistique               &                                                                                 &                                                                                                    & -               \\
\hline

Régression additive généralisée (GAM)              & Régression fortement non linéaire, robuste, locale                                        & Régression statistique               &                                                                                 &                                                                                                    & -               \\
\hline

Régression isotonique                              & Régression non linéaire monotone                                                          & Régression statistique               &                                                                                 &                                                                                                    & Standardisation \\
\hline

Régression Généralisée (GLM)                       & Régression non linéaire                                                                   & Régression statistique               &                                                                                 &                                                                                                    & Standardisation \\
\hline

Quantile regression                                & Régression fortement non linéaire, robuste, présence d'hétéroscédasticité                 & Régression statistique               &                                                                                 & pinball loss                                                                                       & Standardisation \\
\hline

Processus gaussiens (GP)                           & Régression fortement non linéaire                                                         & Régression statistique               &                                                                                 &                                                                                                    & - \\
\hline

Huber regression                                   & Régression fortement non linéaire, robuste, insensible aux outliers                       & Régression statistique               &                                                                                 & huber loss                                                                                         & - \\
\hline

Régression PLS                                     & Régression non linéaire avec analyse des composantes principales                          & Régression non supervisée            &                                                                                 &                                                                                                    & Standardisation \\
\hline

KNN regression                                     & Régression locale / ponctuelle avec beaucoup de données                                   & Régression statistique               & $\hat{y} = \frac{1}{k}\sum_{i \in \mathcal{N}_k(x)} y_i$                        & $\sum_{i=1}^{n}(y_i - \hat{y_i})^2$                                                                & Standardisation \\
\hline

\end{tabularx}
\caption{Synthèse des modèles de Machine Learning - régression}
\end{table}

\newpage

\begin{table}[h!]
\scriptsize
\begin{tabularx}{\textwidth}{|C|C|C|C|C|C|C|}
\hline
\textbf{Modèle}                                    & \textbf{Quand l'utiliser}                                                                 & \textbf{Paradigme}                   & \textbf{Formule}                                                                & \textbf{Quantité à minimiser}                                                                      & \textbf{Préparation des données}  \\
\hline

SVM (Support Vector Machine)                       & Classification fortement non linéaire à marge dure                                        & Classification supervisée            & $\hat{y_i} = \text{sign}({w^{T}x_{i} + b})$                                     & $\frac{1}{2}||w||^2$                                                                               & Standardisation \\
\hline

CSV-C (C-Suport Vector Classification)             & Régression fortement non linéaire avec tolérance aux marges (hyperparamètre non borné)    & Classification supervisée            & $\hat{y_i} = \text{sign}(\sum_{i=1}^{n} \alpha_{i}y_{i}K(x_{i},x_{j}) + b)$     & $\frac{1}{2}||w||^2 + C\sum_{i=1}^{n}\xi_{i}$                                                      & Standardisation \\
\hline

$\nu$-SVM ($\nu$-Suport Vector Machine)            & Régression fortement non linéaire avec tolérance aux marges (hyperparamètre borné)        & Classification supervisée            & $\hat{y_i} = \text{sign}( \sum_{i=1}^{n} \alpha_{i}y_{i}K(x_{i},x_{j}) + b)$    & $\frac{1}{2}||w||^2 - \nu \rho + \frac{1}{n}\sum_{i=1}^{n}\xi_{i}$                                 & Standardisation \\
\hline

SVR / $\epsilon$-SVM (Support Vector Regressor)    & Régression fortement non linéaire                                                         & Régression supervisée                & $\hat{y_i} = \sum_{i=1}^{n}(\alpha_{i} - \alpha_{i}^{*}) K(x_{i},x) + b$        & $\frac{1}{2}||w||^2 + C\sum_{i=1}^{n}(\xi_{i} + \xi_{i}^{*})$                                      & Standardisation \\
\hline

$\nu$-SVR ($\nu$-Support Vector Regressor)         & Régression fortement non linéaire                                                         & Régression supervisée                & $\hat{y_i} = \sum_{i=1}^{n}(\alpha_{i} - \alpha_{i}^{*}) K(x_{i},x) + b$        & $\frac{1}{2}||w||^2 + C(\nu\epsilon + \frac{1}{n}\sum_{i=1}^{n}(\xi_{i} + \xi_{i}^{*}))$           & Standardisation \\
\hline

\end{tabularx}
\caption{Synthèse des modèles de Machine Learning - vecteurs support}
\end{table}

\newpage

\begin{table}[h!]
\scriptsize
\begin{tabularx}{\textwidth}{|C|C|C|C|C|C|C|}
\hline
\textbf{Modèle}                                    & \textbf{Quand l'utiliser}                                                                 & \textbf{Paradigme}                   & \textbf{Formule}                                                                & \textbf{Quantité à minimiser}                                                                      & \textbf{Préparation des données}  \\
\hline

Decision tree                                      & Classification                                                                            & Classification supervisée            & Partitionnement récursif                                                        & $\sum_{i=1}^{n}(y_i - \hat{y_i})²$ dans les feuilles                                               & -               \\
\hline

Random Forest                                      & Classification, robustesse, relations non linéaires                                       & Méthode ensembliste (classification) & Moyenne ou classe majoritaire des arbres                                        & $\sum_{i=1}^{n}(y_i - \hat{y_i})²$ out-of-bag                                                      & -               \\
\hline

Gradient Boosting / XGBoost                        & Classification, robustesse, relations fortement non linéaire                              & Ensemble boosting                    & Actualisation d'un arbre faible                                                 & $-\sum y\log(\hat y)$                                                                              & -               \\
\hline

\end{tabularx}
\caption{Synthèse des modèles de Machine Learning - méthodes d'ensemble}
\end{table}

\newpage

\begin{table}[h!]
\scriptsize
\begin{tabularx}{\textwidth}{|C|C|C|C|C|C|C|}
\hline
\textbf{Modèle}      & \textbf{Quand l'utiliser}                                           & \textbf{Paradigme}                   & \textbf{Formule}                                                                & \textbf{Quantité à minimiser}         & \textbf{Préparation des données}  \\
\hline

AR                   & Séries temporelles                                                  & Modèle autorégressif                 &                                                                                 & -                                     & Stationnarité                      \\
\hline

MA                   & Séries temporelles                                                  & Modèle autorégressif                 &                                                                                 & -                                     & Stationnarité                      \\
\hline

ARMA                 & Séries temporelles                                                  & Modèle autorégressif                 &                                                                                 & -                                     & Stationnarité                      \\
\hline

ARIMA                & Séries temporelles                                                  & Modèle autorégressif                 &                                                                                 & -                                     & Stationnarité                      \\
\hline

SARIMA               & Séries temporelles, saisonnarité                                    & Modèle autorégressif                 &                                                                                 & -                                     & Stationnarité                      \\
\hline

SARIMAX              & Séries temporelles, saisonnarité avec variables exogènes            & Modèle autorégressif                 &                                                                                 & -                                     & Stationnarité                      \\
\hline


\end{tabularx}
\caption{Synthèse des modèles de Machine Learning - processus autorégressifs}
\end{table}

\newpage

\begin{table}[h!]
\scriptsize
\begin{tabularx}{\textwidth}{|C|C|C|C|C|C|C|}
\hline
\textbf{Modèle}      & \textbf{Quand l'utiliser}                                           & \textbf{Paradigme}                   & \textbf{Formule}                                                                & \textbf{Quantité à minimiser}         & \textbf{Préparation des données}  \\
\hline

AR                   & Séries temporelles                                                  & Modèle autorégressif                 &                                                                                 & -                                     & Stationnarité                      \\
\hline


\end{tabularx}
\caption{Synthèse des modèles de Machine Learning - clustering}
\end{table}

\newpage

\begin{table}[h!]
\scriptsize
\begin{tabularx}{\textwidth}{|C|C|C|C|C|C|C|}
\hline
\textbf{Modèle}      & \textbf{Quand l'utiliser}                                           & \textbf{Paradigme}                   & \textbf{Formule}                                                                & \textbf{Quantité à minimiser}         & \textbf{Préparation des données}  \\
\hline

AR                   & Séries temporelles                                                  & Modèle autorégressif                 &                                                                                 & -                                     & Stationnarité                      \\
\hline


\end{tabularx}
\caption{Synthèse des modèles de Machine Learning - réduction de dimension}
\end{table}


\end{document}\documentclass[12pt,french]{report}

% This is the main framework I use for my LateX documents.
% main.tex by Alexis GRACIAS

%%%%%%%%%%%%
% PACKAGES %
%%%%%%%%%%%%

% TABLE OF CONTENT FOR CHAPTERS

\usepackage{minitoc}
\usepackage{minitoc}
\mtcsettitle{minitoc}{Sommaire du chapitre} % Change le titre de la minitoc
\mtcsetfont{minitoc}{section}{\normalsize\bfseries} % Style des sections dans la minitoc
\mtcsetfont{minitoc}{subsection}{\small\itshape} % Style des sous-sections
\mtcsetrules{minitoc}{off} % Désactive les lignes sous les titres de section
\mtcsetoffset{minitoc}{10pt} % Ajoute un espacement autour de la minitoc

\dominitoc % Active les mini-tables des matières
% Personnalisation du titre des mini tables des matières
\renewcommand{\mtctitle}{}


% GEOMETRY

\usepackage[left=2cm,right=2cm,top=2cm,bottom=2cm]{geometry}

% LASTAPGE

\usepackage{lastpage}

% HEADERS AND FOOTERS
\usepackage{fancyhdr}

% LANGUAGE

\usepackage{babel}

% REFERENCES

\usepackage{hyperref} % Add links to the table of content, files and website

% GRAPHS

\usepackage{tikz} % Graph package

% IMAGES

\usepackage{graphicx} % Required for inserting images
\usepackage{tabto}

% COLOR

\usepackage[skins,theorems]{tcolorbox} % Frame color package
\usepackage{amssymb}
\usepackage[dvipsnames]{xcolor} % Text color package and more colors
\usepackage{tcolorbox}

% Custom colors
\definecolor{ao}{rgb}{0.0, 0.0, 1.0}
\definecolor{coolblack}{rgb}{0.0, 0.18, 0.39}
\definecolor{cyan}{rgb}{0.0, 1.0, 1.0}
\definecolor{glaucous}{rgb}{0.38, 0.51, 0.71}
\definecolor{electricultramarine}{rgb}{0.25, 0.0, 1.0}

% CAPTIONS

\usepackage{caption}
\usepackage{subcaption}
\usepackage[utf8]{inputenc}

% STYLE

% Chapter's title style
\usepackage[T1]{fontenc}
\usepackage{titlesec, blindtext, color}
\definecolor{gray75}{gray}{0.75}
\newcommand{\hsp}{\hspace{20pt}}
\titleformat{\chapter}[hang]{\Huge\bfseries}{\thechapter\hsp\textcolor{gray75}{|}\hsp}{0pt}{\Huge\bfseries}

% Table of contents style
\makeatletter
\renewcommand*{\l@chapter}[2]{%         % Chapter label format
  \ifnum \c@tocdepth > \m@ne            % Check: Maximum depth > -1
    \addpenalty{-\@highpenalty}%        % Encourage prior line breaking
    \vskip 1.0em \@plus\p@%             % Prior vertical whitespace
    \setlength\@tempdima{1.5em}%        % Set temporary length
    \begingroup                         % Begin closed group
      \parindent\z@                     % No paragraph indentation
      \rightskip\@tocrmarg              % Reserved right whitespace
      \parfillskip -\@tocrmarg          % Prevent box from moving
      \leavevmode                       % Force horizontal mode
      \advance\leftskip\@tempdima       % Left whitespace
      \hskip -\leftskip                 % First line alignment
      {\bfseries #1}%           % Chapter name and format
      \nobreak                          % Avoid lineskip
      \leaders\hbox{$%                  % Command to replicate dots
        \m@th                           % Math mode for alignment
        \mkern\@dotsep mu               % Whitespace before a dot
        \hbox{.}%                       % Dot separator
        \mkern\@dotsep mu               % Whitespace after a dot
      $}%
      \hfill                            % Flush to the right
      \nobreak\hb@xt@\@pnumwidth{%      % Chapter page’s number box
        \hss #2%                        % Page number with buffer
      }%
      \par                              % Insert line skip
      \penalty\@highpenalty             % Return to default penalty
    \endgroup                           % End closed group
  \fi
}
\makeatother


% MATH

\usepackage{amsmath,amsfonts,mathtools,stmaryrd} % Math libraries

\usepackage{listings} % required for specific languages
\lstset{ % Set listing package options
    language=bash, % choose the language of the code
    basicstyle=\fontfamily{pcr}\selectfont\footnotesize\color{red},
    keywordstyle=\color{black}\bfseries, % style for keywords
    numbers=none, % where to put the line-numbers
    numberstyle=\tiny, % the size of the fonts that are used for the line-numbers     
    backgroundcolor=\color{white},
    showspaces=false, % show spaces adding particular underscores
    showstringspaces=false, % underline spaces within strings
    showtabs=false, % show tabs within strings adding particular underscores
    frame=single, % adds a frame around the code
    tabsize=2, % sets default tabsize to 2 spaces
    rulesepcolor=\color{gray},
    rulecolor=\color{black},
    captionpos=b, % sets the caption-position to bottom
    breaklines=true, % sets automatic line breaking
    breakatwhitespace=false, 
}

% Tabular

\usepackage{tabularx}
\usepackage{ragged2e}

\newcolumntype{L}{>{\RaggedRight\arraybackslash}X} 
\newcolumntype{C}{>{\centering\arraybackslash}X}

% OTHER

\usepackage{pdfpages}

% SCRIPTS (python)

% Load libraries
\usepackage{tcolorbox}
\tcbuselibrary{minted,breakable,xparse,skins}
\usepackage{minted}
\usepackage{xcolor}

\definecolor{bg}{gray}{0.95}
\DeclareTCBListing{mintedbox}{O{}m!O{}}{%
  breakable=true,
  listing engine=minted,
  listing only,
  minted language=#2,
  minted style=default,
  minted options={%
    linenos,
    gobble=0,
    breaklines=true,
    breakafter=,,
    fontsize=\small,
    numbersep=8pt,
    #1},
  boxsep=0pt,
  left skip=0pt,
  right skip=0pt,
  left=25pt,
  right=0pt,
  top=3pt,
  bottom=3pt,
  arc=5pt,
  leftrule=0pt,
  rightrule=0pt,
  bottomrule=2pt,
  toprule=2pt,
  colback=bg,
  colframe=gray!70,
  enhanced,
  overlay={%
    \begin{tcbclipinterior}
    \fill[gray!20!white] (frame.south west) rectangle ([xshift=20pt]frame.north west);
    \end{tcbclipinterior}},
  #3}

%\usepackage{eso-pic,lipsum}
%\AddToShipoutPicture{%
%	\AtTextCenter{%
%		\fboxsep5mm \fboxrule=0.8pt
%		\makebox(0,0)[c]{\fbox{\rule{0pt}\textheight\rule\textwidth{0pt}}}%
%	}%
%}

%%%%%%%%%%%%
% COMMANDS %
%%%%%%%%%%%%

\lstset{aboveskip=\baselineskip,belowskip=\baselineskip,basicstyle=\ttfamily} % Formating line break after bash commands

% Get rid of 0. chapter's number
\makeatletter 
\renewcommand{\thesection}{%
  \ifnum\c@chapter<1 \@arabic\c@section
  \else \thechapter.\@arabic\c@section
  \fi
}
\makeatother

% Rules for \bar{x} and \overline{x} commands
\makeatletter
\newcommand*{\Xbar}{}%
\DeclareRobustCommand*{\Xbar}{%
  \mathpalette\@Xbar{}%
}


% THEOREMS

\definecolor{electricultramarine}{rgb}{0.25, 0.0, 1.0}

\tcbuselibrary{theorems} 
\newtcbtheorem
  []% init options
  {Syntax}% name
  {Syntax}% title
  {%
    colback=BrickRed!2,
    colframe=BrickRed!35!BrickRed,
    fonttitle=\bfseries,
  }% options
  {def}% prefix

\tcbuselibrary{theorems} 
\newtcbtheorem
  []% init options
  {Input}% name
  {Input}% title
  {%
    colback=BlueViolet!5,
    colframe=BlueViolet!35!BlueViolet,
    fonttitle=\bfseries,
  }% options
  {def}% prefix

\tcbuselibrary{theorems} 
\newtcbtheorem
  []% init options
  {Output}% name
  {Output}% title
  {%
    colback=Emerald!5,
    colframe=Emerald!35!Emerald,
    fonttitle=\bfseries,
  }% options
  {def}% prefix

% FORMULAS

\tcbset{highlight math style={enhanced,
  colframe=red,colback=white,arc=0pt,boxrule=1pt}}

% CUSTOM COMMANDS

\newcommand*{\@Xbar}[2]{%
  % #1: math style
  % #2: unused (empty)
  \sbox0{$#1\mathrm{X}\m@th$}%
  \sbox2{$#1X\m@th$}%
  \rlap{%
    \hbox to\wd2{%
      \hfill
      $\overline{%
        \vrule width 0pt height\ht0 %
        \kern\wd0 %
      }$%
    }%
  }%
  \copy2 %
}
\makeatother

%%%%%%%%%%%%%%%%%%%%%%%%%%%%%%
% DECLARE UNICODE CHARACTERS %
%%%%%%%%%%%%%%%%%%%%%%%%%%%%%%

% Main
\DeclareUnicodeCharacter{2500}{\textendash}  % ─
\DeclareUnicodeCharacter{2502}{\textbar}     % │
\DeclareUnicodeCharacter{250C}{\textup{┌}}   % ┌
\DeclareUnicodeCharacter{2510}{\textup{┐}}   % ┐
\DeclareUnicodeCharacter{2514}{\textup{└}}   % └
\DeclareUnicodeCharacter{2518}{\textup{┘}}   % ┘
\DeclareUnicodeCharacter{251C}{\textup{├}}   % ├
\DeclareUnicodeCharacter{2524}{\textup{┤}}   % ┤
\DeclareUnicodeCharacter{252C}{\textup{┬}}   % ┬
\DeclareUnicodeCharacter{2534}{\textup{┴}}   % ┴
\DeclareUnicodeCharacter{253C}{\textup{┼}}   % ┼
\DeclareUnicodeCharacter{2550}{\textup{═}}   % ═
\DeclareUnicodeCharacter{2551}{\textup{║}}   % ║
\DeclareUnicodeCharacter{2552}{\textup{╒}}   % ╒
\DeclareUnicodeCharacter{2553}{\textup{╓}}   % ╓
\DeclareUnicodeCharacter{2554}{\textup{╔}}   % ╔
\DeclareUnicodeCharacter{2555}{\textup{╕}}   % ╕
\DeclareUnicodeCharacter{2556}{\textup{╖}}   % ╖
\DeclareUnicodeCharacter{2557}{\textup{╗}}   % ╗
\DeclareUnicodeCharacter{2558}{\textup{╘}}   % ╘
\DeclareUnicodeCharacter{2559}{\textup{╙}}   % ╙
\DeclareUnicodeCharacter{255A}{\textup{╚}}   % ╚
\DeclareUnicodeCharacter{255B}{\textup{╛}}   % ╛
\DeclareUnicodeCharacter{255C}{\textup{╜}}   % ╜
\DeclareUnicodeCharacter{255D}{\textup{╝}}   % ╝
\DeclareUnicodeCharacter{2560}{\textup{╠}}   % ╠
\DeclareUnicodeCharacter{2561}{\textup{╡}}   % ╡
\DeclareUnicodeCharacter{2562}{\textup{╢}}   % ╢
\DeclareUnicodeCharacter{2563}{\textup{╣}}   % ╣
\DeclareUnicodeCharacter{2564}{\textup{╤}}   % ╤
\DeclareUnicodeCharacter{2565}{\textup{╥}}   % ╥
\DeclareUnicodeCharacter{2566}{\textup{╦}}   % ╦
\DeclareUnicodeCharacter{2567}{\textup{╧}}   % ╧
\DeclareUnicodeCharacter{2568}{\textup{╨}}   % ╨
\DeclareUnicodeCharacter{2569}{\textup{╩}}   % ╩
\DeclareUnicodeCharacter{256A}{\textup{╪}}   % ╪
\DeclareUnicodeCharacter{256B}{\textup{╫}}   % ╫
\DeclareUnicodeCharacter{256C}{\textup{╬}}   % ╬

% Blocks and bars
\DeclareUnicodeCharacter{2580}{\textup{▀}}   % ▀
\DeclareUnicodeCharacter{2584}{\textup{▄}}   % ▄
\DeclareUnicodeCharacter{2588}{\textup{█}}   % █
\DeclareUnicodeCharacter{258C}{\textup{▌}}   % ▌
\DeclareUnicodeCharacter{2590}{\textup{▐}}   % ▐
\DeclareUnicodeCharacter{2591}{\textup{░}}   % ░
\DeclareUnicodeCharacter{2592}{\textup{▒}}   % ▒
\DeclareUnicodeCharacter{2593}{\textup{▓}}   % ▓
\DeclareUnicodeCharacter{25A0}{\textup{■}}   % ■
\DeclareUnicodeCharacter{25A1}{\textup{□}}   % □
\DeclareUnicodeCharacter{25AA}{\textup{▪}}   % ▪
\DeclareUnicodeCharacter{25AB}{\textup{▫}}   % ▫

% Geometry
\DeclareUnicodeCharacter{25B6}{\textup{▶}}   % ▶
\DeclareUnicodeCharacter{25C0}{\textup{◁}}   % ◁
\DeclareUnicodeCharacter{25FB}{\textup{▫}}   % ▫
\DeclareUnicodeCharacter{25FD}{\textup{▽}}   % ▽
\DeclareUnicodeCharacter{25FE}{\textup{▾}}   % ▾

% Currency
\DeclareUnicodeCharacter{20AC}{\textup{€}}   % €
\DeclareUnicodeCharacter{00A3}{\textup{£}}   % £
\DeclareUnicodeCharacter{20B9}{\textup{₹}}   % ₹

% Arrows
\DeclareUnicodeCharacter{2190}{\textup{←}}   % ←
\DeclareUnicodeCharacter{2191}{\textup{↑}}   % ↑
\DeclareUnicodeCharacter{2192}{\textup{→}}   % →
\DeclareUnicodeCharacter{2193}{\textup{↓}}   % ↓
\DeclareUnicodeCharacter{21D0}{\textup{⇐}}   % ⇐
\DeclareUnicodeCharacter{21D1}{\textup{⇑}}   % ⇑
\DeclareUnicodeCharacter{21D2}{\textup{⇒}}   % ⇒
\DeclareUnicodeCharacter{21D3}{\textup{⇓}}   % ⇓

% Miscellaneous
\DeclareUnicodeCharacter{2605}{\textup{★}}   % ★
\DeclareUnicodeCharacter{2606}{\textup{☆}}   % ☆
\DeclareUnicodeCharacter{2610}{\textup{☐}}   % ☐
\DeclareUnicodeCharacter{2611}{\textup{☑}}   % ☑
\DeclareUnicodeCharacter{2612}{\textup{☒}}   % ☒
\DeclareUnicodeCharacter{2620}{\textup{☠}}   % ☠
\DeclareUnicodeCharacter{2622}{\textup{☢}}   % ☢
\DeclareUnicodeCharacter{2623}{\textup{☣}}   % ☣
\DeclareUnicodeCharacter{2626}{\textup{☦}}   % ☦
\DeclareUnicodeCharacter{262A}{\textup{☪}}   % ☪
\DeclareUnicodeCharacter{262F}{\textup{☯}}   % ☯
\DeclareUnicodeCharacter{2660}{\textup{♠}}   % ♠
\DeclareUnicodeCharacter{2665}{\textup{♥}}   % ♥
\DeclareUnicodeCharacter{2666}{\textup{♦}}   % ♦
\DeclareUnicodeCharacter{2663}{\textup{♣}}   % ♣

%%%%%%%%%%%%%%%%%%%%%%
% DOCUMENTS SETTINGS %
%%%%%%%%%%%%%%%%%%%%%%

% Fist page
\newcommand{\HRule}{\rule{\linewidth}{0.5mm}}

\title{
    \HRule \\[0.4cm] \Huge \bfseries 
    Machine learning cheatsheet 
    \\[0.15cm] \HRule \\[0.4cm]}
\author{\Large Alexis GRACIAS}


% Document
\begin{document}


\maketitle

\dominitoc % Initialization
\large \tableofcontents
\normalsize % Text size

% Headers and footers
\renewcommand{\chaptermark}[1]{ \markboth{#1}{} }
\pagestyle{fancy}
%... then configure it.
\fancyhead{} % clear all header fields
\fancyhead[RO,LE]{\leftmark}
\fancyhead[LO,LE]{A. GRACIAS}
\fancyfoot{} % clear all footer fields

% Page number
\makeatletter
\renewcommand{\@evenfoot}{\makebox[\textwidth][c]{page {\thepage} sur \pageref{LastPage}}}
\renewcommand{\@oddfoot}{\@evenfoot}
\makeatother

% ITEMIZE (lists)
\renewcommand{\labelitemi}{$\bullet$}
\renewcommand{\labelitemii}{---}
\renewcommand{\labelitemiii}{-}
\renewcommand{\labelitemiv}{.}

%%%%%%%%%%%
% INCLUDE %
%%%%%%%%%%%

\include{content}


\end{document}\input{main}